\documentclass[11pt]{article}
%  \usepackage{evan}
\usepackage{mathtools}
\usepackage{sagetex}
\usepackage{amsfonts}
\usepackage{amsmath}
%\usepackage{sectsty}
% \usepackage{graphicx}
\newtheorem{thm}{Theorem}
\newtheorem{defn}{Definition}
\newtheorem{ex}{Example}
\newcommand{\Bold}[1]{#1}
% Margins
\topmargin=-0.45in
\evensidemargin=0in
\oddsidemargin=0in
\textwidth=6.5in
\textheight=9.0in
\headsep=0.25in

\title{Linear Algebra Notes}
\author{Atticus Kuhn}
\date{\today}


\begin{document}
\maketitle
\tableofcontents
\section{March 20}
\begin{defn}
  A \textbf{linear equation} of the variables $x_{1}, x_{2}, \ldots x_{n}$ is any equation of the form
  \[a_{1}x_{1}+a_{2}x_{2} + \cdots + a_{n}x_{n} = b\]
  where $a_{1}, \ldots a_{n}$ are called the coefficents and $b$ is called the constant.
  \end{defn}
  \begin{defn}
    A \textbf{system of linear equation} is a set of 1 or more linear equations.
    \end{defn}
    \begin{ex}
      \[x + 2y = 3\]
      is a linear system and

      \begin{align*}
        x-2y = 1 \\
        2x + 3y = 4 \\
        x + y =  5
        \end{align*}
        is a linear system.
        But
        \[5 \sqrt{x} - y=1\]
        is not a linear system.
      \end{ex}

      Given a linear system, we want to find all substitutions which satisfy the linear
      system.
      A \textbf{solution set} is a set of n-tuples, all of which are valid solutions to a given system of linear equations.
      Our central problem is: given a linear system, find its solution set.
      Big problems
      \begin{enumerate}
        \item Existance Question: Do there exist any solutions? If there are, we call the system \textbf{consistent}. If there are no solutions, we call the solution \textbf{inconsistent}.
              \item Uniqueness Question: How many solutions are there? If there is exactly 1 solution, we call that \textbf{unique}.
      \end{enumerate}
      \begin{thm}
        If a linear system has solutions, it either has 1 solution or infintely many solutions.
        A linear system cannot have 12 solutions.
      \end{thm}
      \begin{ex}
        \[x + 2y = 5, x + 4y = 6\]
        has solution
        \[x = -4, y=9/2\]
        This linear system corresponds to the matrix multiplication
        \[\begin{bmatrix} 1 & 2 \\ 3 & 4 \end{bmatrix} \begin{bmatrix} x \\ y \end{bmatrix} = \begin{bmatrix} 5 \\ 6 \end{bmatrix}\]
        We can turn this into an \textbf{augmented matrix} to get
        \[
          \begin{bmatrix}
            1 & 2 & 5 \\
            3 & 4 & 6
            \end{bmatrix}
        \]
        We are allowed to perform row operations on the rows to get
\[
          \begin{bmatrix}
            1 & 2 & 5 \\
            0 & 1 & 9/2
          \end{bmatrix}
          \implies
          \begin{bmatrix}
            1 & 0 & -4\\
            0 & 1 & 9/2
            \end{bmatrix}
        \]
        So $(x,y) = (-4, 9/2)$. The solution is unique.
      \end{ex}

      \begin{thm}
        We are allowed to do these elementary row operations
        \begin{defn}
          We say two systems of equations are \textbf{equivalent}
          if their solutions sets are equal.
          \end{defn}
        \begin{enumerate}
          \item replacement: replace one row with the sum of itself and another row
          \item interchange: swap 2 rows
                \item scaling: multiply all entries in a  row by a constant
                \end{enumerate}
        \end{thm}
        The reason the elementary operations work is because they
        give an equivalent system, i.e. they do not change the solution set.

        Convince yourself that each of the elementary operations do not change the solution
        set.
        \begin{ex}
          Let's solve this system
          \begin{align*}
            kx + y + 2z &=1 \\
            x + z &= h \\
            y-z &= 1
            \end{align*}
            The solution is
            \[x=-3h/(k-3), y=\frac{(h+1)k-3}{k-3}, z = \frac{hk}{k-3}\]
            The system would have no solutions (inconsistent) iff $k=3$ and $h \neq 0$.
            The system has infinite solutions if $k=3$ and $h= 0$.
            The system has 1 solution if $k \neq 3$.
          \end{ex}
          \begin{defn}
            A matrix is in \textbf{echelon form} iff
            \begin{enumerate}
              \item All nonzero rows are above any rows of all zeroes
              \item each leading entry in a row is in a column to the right of the leading
                    endtry of the row above it
                \item All entries in a column below a leading entry are zeroes
            \end{enumerate}
          \end{defn}
          \begin{defn}
            A matrix is in \textbf{reduce row echelon form} iff
            \begin{enumerate}
              \item The leading entry in each nonzero row is $1$
                    \item each leading $1$ is the only nonzero entry in its column
                    \end{enumerate}
            \end{defn}
            \begin{ex}
              This matrix
              \[
                \begin{bmatrix}
                  * & * & 0 & 0 & 0 & 0 \\
                  0 & * & 1 & 0 & 0 & 0 \\
                  0 & 0 & * & 0 & 0 & *
                  \end{bmatrix}
                \]
                is in REF, but not in RREF.
              \end{ex}
              \begin{ex}
                This matrix
                \[\begin{bmatrix}
                    0 & 1  & 0 & 0 & 0 \\
                    0 & 0 & 2 & 0 & 0 \\
                    0 & 0 & 0 & 0 & 1 \\
                    0 & 0 & 0 & 0 & 0
                    \end{bmatrix}\]
                  is in REF, and it would be in RREF if we changed the $2$ to a $1$.
                \end{ex}

                \begin{thm}
                A given matrix has infinitely many equivalent REF, but only 1 RREF.
                \end{thm}
                \begin{ex}
                  Reduce this matrix
                  \[
                    \begin{bmatrix}
                      1 & -1 & -1 \\
                      0 & 2 & 1 \\
                      2 & 0 & -1
                      \end{bmatrix}
                  \]
                  Solution:
                  \[
\left(\begin{array}{rrr}
1 & 0 & -\frac{1}{2} \\
0 & 1 & \frac{1}{2} \\
0 & 0 & 0
\end{array}\right)
                  \]
   \end{ex}
 \begin{defn}
   If a matrix is in REF, the \textbf{pivots}
   are leading entries. The \textbf{pivot columns} are
   the columns which contain a pivot.
\end{defn}
\begin{defn}
  A \textbf{basic variable} is a variable which corresponds to a pivot column.
  A \textbf{free variable} is a variable which does not correspond to a pivot column.
\end{defn}
\begin{ex}
  Given the matrix
  \[\begin{bmatrix}
      1 & 0 & 2 & 0 & 4 \\
      0 & 1 & 3 & 0 & 5 \\
      0 & 0 & 0 & 1 & 6 \\
    0 & 0 & 0 & 0 & 0 \end{bmatrix}\]
This matrix is in RREF. Variables $x_{1}, x_{2}, x_{4}$ are basic variables
and variable $x_{3}$ is a free variable.
\end{ex}

If the righmost column of $A$ is a pivot column, then we cannot tell if $\mathbf{A}\vec{x} = \vec{b}$ is consistent or inconsistent.
\begin{thm}
  If the rightmost column of
  $\left[\mathbf{A} \quad \vec{b}\right]$ is a pivot column, then
  $\mathbf{A}\vec{x} = \vec{b}$ is not consistent.
\end{thm}

\begin{thm}
  Even if you have a free variable, that doesn't guarantee infintely many solutions.

  If you have a free variable and the system is consistent, then the system has infinitely many solutions.
\end{thm}

\begin{ex}
  Given that the RREF is
  \[\begin{bmatrix}
      1 & 2 & 3 & 0 &  0 &  6 \\
      0 & 0 & 0  & 1 & 0 & 7 \\
      0 & 0 & 0 & 0 & 1 & 8\end{bmatrix}\]
  Describe the solution set.
  It is
  \[
    \begin{bmatrix}
      x_{1} \\ x_{2} \\ x_{3} \\ x_{4} \\ x_{5}
    \end{bmatrix}
    =
    x_{2} \begin{bmatrix} -2  \\ 1 \\ 0 \\ 0 \\ 0 \end{bmatrix}
    +
    x_{3} \begin{bmatrix} -3 \\ 0 \\ 1 \\ 0 \\ 0 \end{bmatrix}
    +
    \begin{bmatrix} 6 \\ 0 \\ 0 \\ 7 \\ 8 \end{bmatrix}
  \]
  This solution set is infinitely big because it has free varaibles.
  This is called \textbf{parametric vector form} of the solution set.
\end{ex}
\section{March 22}
Let's review what we learned last session about matricies and linear systems.
\begin{enumerate}
  \item the RREF form of a matrix is unique (but the REF form is not).
  \item the elementary row operations do not affect the solution set of a linear system.
  \item pivot columns correspond to basic variables, non-pivot columns correpond to free variables.
  \item every linear system has either 0, 1, or $\infty$ solutions.
  \item an augmented matrix can be used to represent a linear system. It is the coefficent matrix concatenated with the constant vector.
  \item We can express the solution set of a linear system in parametric vector form.
  \item If a system is consistent and has a free variable, it has infinitely many variables (but even inconsistent systems may have free variables)
  \item row operations do not change pivot columns.
\end{enumerate}

\begin{ex}
  Let
  \[p = (0,1) \qquad q = (1,1) \qquad r = (2,2)\]
  Prove there exists no line with all 3 points.

  Prove there exists exactly one unique parabola with all 3 points
  \\
  Solution: \\

  The line has linear system
  \[1  = 0a + b \qquad 1 = a + b \qquad 2 = 2a + b\]
  \begin{align*}
    \begin{bmatrix}
      0 & 1 & 1 \\
      1 & 1 & 1 \\
      2 & 1 & 2 \\
    \end{bmatrix}
    \sim
\left(\begin{array}{rrr}
1 & 0 & 0 \\
0 & 1 & 0 \\
0 & 0 & 1
\end{array}\right)
    \end{align*}
\end{ex}
So inconsistent.

For the parabola, it has the system
\[1 = 0a+0b+c \qquad 1 = a + b +c \qquad 2 = 4a + 2b + c\]
\begin{align*}
  \begin{bmatrix}
    0 & 0 & 1 & 1 \\
    1 & 1 & 1 & 1\\
    4 & 2 & 1 & 2 \\
  \end{bmatrix}
  \sim \\
\left(\begin{array}{rrrr}
1 & 0 & 0 & \frac{1}{2} \\
0 & 1 & 0 & -\frac{1}{2} \\
0 & 0 & 1 & 1
\end{array}\right)
\end{align*}
i.e.
\[y=\frac{1}{2}x^{2} - \frac{1}{2}x + 1\]
solution is unique because no free varaibles.
\subsection{1.3 -- Vector Equation}

\begin{defn}
  \textbf{matrix multiplication by vector}: Given
  \[A = [\vec{a_{1}}, \vec{a_{2}}, \vec{a_{3}}, \ldots, \vec{a_{n}}]\]
  and
  \[\vec{v} = \begin{bmatrix} v_{1} \\ v_{2} \\ v_{3} \\ \vdots \\ v_{n} \end{bmatrix}\]
  then
  \[A\vec{v} = \vec{a_{1}}v_{1}+\vec{a_{2}}v_{2} + \cdots + \vec{a_{n}}v_{n} = \sum_{i=1}^{n} \vec{a_{i}}v_{i}\]
  We call this a \textbf{linear combination} of $a_{i}$. A linear combination is a
  sum of scalings of vectors, i.e. scale the vectors and then add them up.
\end{defn}

\begin{ex}
  \[\begin{bmatrix} 1 & 2 & 3 \\ 5 & 4 & 7 \end{bmatrix} \begin{bmatrix} a \\ b \\ c \end{bmatrix} = \begin{bmatrix} a + 2b + 3c \\ 5a + 4b + 7c \end{bmatrix}\]
\end{ex}

\begin{defn}
  \textbf{span}: Given a set of vectors $S = \{\vec{v_{1}}, \vec{v_{2}}, \ldots, \vec{v_{n}} \}$, then $span(S)$ is the set of
  all linear combinations of the vectors. In other words, it is the set of all vectors reachable by using only the vectors in $S$.
  In equational form,
  \[span(S) = \left\{ \left .\sum_{i=1}^{n} c_{i} \vec{v}_{i} \;\right |\; \forall c_{i} \in \mathbb{R}\right\}\]
\end{defn}


\begin{ex}
  In $\mathbb{R}^{3}$,
  \[span(\{\vec{v}\}) = \{k\vec{v}  \quad | \quad k \in \mathbb{R}\}\]
  This set is a line going through the origin (unliness $\vec{v} = \vec{0}$, then it's just the origin).
  The $\vec{0}$ vector is always in the span set.

  \[span(\{\vec{u}, \vec{v}\})\]
  this is a plane passing through the origin (unless the vectors are linearly dependent, then it's a line). It could be origin, or line thru
  origin, or plane thru origin.

  \[span(\{\vec{u}, \vec{v}, \vec{w}\})\]
  is either the origin, or a line thru the origin, or a plane thru the origin, or the entire space.
\end{ex}

\begin{ex}
  \[A = \begin{bmatrix} 1  & -1 \\ 2 & 1 \\ 1 & 0 \end{bmatrix}, \qquad v = \begin{bmatrix} 0 \\ 1 \\ 2 \end{bmatrix}\]
  Is $Ax = b$ consistent?
  This is equivalent to asking if $b$ is a linear combination of the columns of $A$ and it is equivalent to asking
  if $b$ is an element of the spanning set of the columns of $A$.

  This is not consistent because
  \[\left(\begin{array}{rrr}
1 & -1 & 0 \\
2 & 1 & 1 \\
1 & 0 & 2
\end{array}\right)\sim \left(\begin{array}{rrr}
1 & 0 & 0 \\
0 & 1 & 0 \\
0 & 0 & 1
\end{array}\right)\]

Thus, $b \not\in span(A)$
\end{ex}

\begin{ex}
  \[x + 2y + 3z = 4 \qquad 5x + 6y + 7z = 8 \]
  So
  \[\begin{bmatrix} 1 & 2 & 3 \\ 5 & 6 & 7 \end{bmatrix} \begin{bmatrix} x \\ y \\ z \end{bmatrix} = \begin{bmatrix} 4 \\ 8 \end{bmatrix}\]
\end{ex}

\begin{thm}
  Asking whether the system $Ax = b$ is consistent is the same as asking if $b \in span(A)$
\end{thm}
\section{March 23}
\subsection{1.4 -- Matrix Equation}

\begin{ex}
  Let
  \[A =\begin{bmatrix} 1 & 4 & 5 \\ -3 & -1 & -14 \\ 2 & 8 & 10 \end{bmatrix}\]
  Let $b \in \mathbb{R}^{m}$. Is $Ax=b$ consistent for any $b$? \\
  Solution: \\
  Not consistent because if you rref
  \[
\left(\begin{array}{rrrr}
1 & 4 & 5 & b_{1} \\
-3 & -11 & -14 & b_{2} \\
2 & 8 & 10 & b_{3}
\end{array}\right) \]
you get the $-2b_{1} + b_{3} = 0$.

Now, find the set spanned by the column vectors of $A$. \\

(note that the spanning set is NOT $\mathbb{R}^{3}$, because the columns have a
rank of 2).
This is
\[span(A) = \left \{\left . \begin{bmatrix} b_{1} \\ b_{2} \\ b_{3} \end{bmatrix} \; \right\vert -2b_{1} + b_{3} = 0\right\} = \left\{ b_{2} \begin{bmatrix} 0 \\ 1 \\ 0 \end{bmatrix} + b_{3} \begin{bmatrix} 1 \\ 0 \\ 2 \end{bmatrix} \vert \forall b_{2}, b_{3} \in \mathbb{R}\right\} = span\left(\left\{\begin{bmatrix} 0 \\ 1 \\ 0 \end{bmatrix}, \begin{bmatrix} 1 \\ 0 \\2\end{bmatrix}\right\}\right)\]
You can do this is Sage with
\begin{sageblock}
 A =  span(column_matrix([[1,4,5],[3,-11,-14], [2,8,10]]))
\end{sageblock}
\end{ex}
\[A = \sage{A}\]
\begin{thm}
  The following statements are logically equivalent:
  Let $A$ be an $m \times n$ matrix and let $b$ be an $m \times 1$ vector.
  \begin{enumerate}
      \item for all $b \in \mathbb{R}^{m}$, $Ax = b$ has a solution.
      \item $span(A) = \mathbb{R}^{m}$
    \item The columns of $A$ span $\mathbb{R}^{m}$
      \item $A$ has 1 pivot position in every row.
      \item $b$ can be written as a linear combination of the columns of $A$.
      \item $Ax = b$ is consistent for all $b \in \mathbb{R}^{m}$.
  \end{enumerate}
\end{thm}

Make sure you don't mess up the details. We want a pivot in every ROW of $A$, not every column of $A$.
\begin{ex}
  Let's apply theorem 7.
  \[span\left(\left\{ \begin{bmatrix} 1 \\ 2 \end{bmatrix}\right\}\right) \neq \mathbb{R}^{2}\]
  because it doesn't have a pivot in the second row.
\end{ex}

\begin{ex}
  \[span\left(\left\{ \begin{bmatrix} 1 \\ 1 \end{bmatrix},\begin{bmatrix} 1 \\ 0 \end{bmatrix},\begin{bmatrix} 2 \\ 1 \end{bmatrix}\right\}\right) = \mathbb{R}^{2}\]
  because there's a pivot in every row.
  Note that there is not a pivot in every column, but the theorem requires a pivot in every row.
\end{ex}

Another mistake people make is thinking that the theorem applies if there is a pivot in every row of $\left [ A \quad b\right ]$. We are only concerned
about a pivot in every row of $A$.
\section{March 27}
\subsection{Homogenous System}
\begin{defn}
  The linear system $Ax=b$ is called \textbf{homogenous} iff $b=\vec{0}$. A homogenous linear system is any system of the form $Ax = \vec{0}$.
\end{defn}
\begin{thm}
  A homogenous linear system is always consistent because \[\mathbf{A}\vec{0} = \vec{0}\]
\end{thm}
\begin{defn}
  We call $x = \vec{0}$ the \textbf{trivial solution} to the linear system $Ax = \vec{0}$.
\end{defn}
\begin{thm}
  A homogenous linear system has infinite nontrivial solutions if and only if it contains free variables.
\end{thm}
\begin{ex}
  consider this homogenous linear system
  \[\begin{bmatrix} 1 & 2 \\ 3  & 6 \end{bmatrix} \begin{bmatrix} x \\ y \end{bmatrix} = \begin{bmatrix} 0 \\ 0 \end{bmatrix}\]
  The solution set is
  \[span\left\{\begin{bmatrix} -2 \\ 1 \end{bmatrix}\right\}\]
\end{ex}

\begin{ex}
  consider this homogenous linear system
  \[\begin{bmatrix} 1 & 2 \\ 3  & 6 \end{bmatrix} \begin{bmatrix} x \\ y \end{bmatrix} = \begin{bmatrix} 4 \\ 12 \end{bmatrix}\]
  Then
  \[\begin{bmatrix} x \\ y \end{bmatrix} = \begin{bmatrix} -2y + 4 \\ y \end{bmatrix}\]
  This solution set cannot be written as a span because it doesn't contain $\vec{0}$.
\end{ex}

The solution set of a homogenous system can always be expressed as the span of some vectors.

\begin{ex}
  Here are 2 systems to consider.
  \[S_{1} : x + 2y + 3z = 0\]
  \[S_{2} : x + 2y + 3z = 4 \]
  Compare and contrast the solution sets. \\
  Solution: \\
  The solution set for $S_{1}$ is $span \left\{\begin{bmatrix}-3 \\ 0 \\ 1\end{bmatrix}, \begin{bmatrix} -2 \\ 1 \\ 0 \end{bmatrix}\right\}$.
  The solution set for $S_{2}$ is
  \[\left \{\left.\begin{bmatrix} -2y-3z + 4 \\ y \\ z \end{bmatrix} \right| \forall y,z \in \mathbb{R}\right\}\]
\end{ex}

\begin{thm}
  The solution set to $Ax = b$ is parallel to the solution set of $Ax = 0$. You can think of the solution set of $Ax=b$ as being
  a constant vector translation from the solution set of $Ax=0$.

  Suppose $p$ is any solution to the system $\mathbf{A}\vec{x} = \vec{b}$. Then, any other solution to $\mathbf{A}\vec{x} = \vec{b}$ can
  be written as $\vec{p} + \vec{v}_{h}$, where $\vec{v}_{h}$ is any solution to $\mathbf{A}\vec{x} = \vec{0}$.
\end{thm}
Warning: you can only think of the solution set of $\mathbf{A}\vec{x} = \vec{b}$ as a translation of the solution set of $\mathbb{A}\vec{x} = \vec{0}$ is the linear system is consistent. Otherwise, it's solution set is the
empty set.
\subsection{1.6 -- Applications}
There are a bunch of applications of matricies in the textbook.
\begin{ex}
  Find the general traffic pattern in the freeway network shown in the figure \\
  Describe the general traffic pattern when the road $x_{1}$ is closed. \\
  When $x_{4} = 0$ what is the minimum value of $x_{1}$? \\
  Solution:\\
  For the solution, we use the general principle that (flow in) - (flow out) = 0.
  \[A = \begin{bmatrix} 1 & 0 & -1 & -1 & 0 & 40 \\
          -1 & -1 & 0 & 0 & 0 & -200 \\
          0 & 1 & 1 & 0 & -1 & 100 \\
          0 & 0 & 0 & 1 & 1 & 60\end{bmatrix}\]
\begin{align*}
  A &: x_{1} &= x_{3} + x_{4} + 40\\
  B &: 200 &= x_{1} + x_{2} \\
  C &: x_{2} + x_{3} &= 100 + x_{5} \\
  D &: x_{4} + x_{5} &= 60 \\
\end{align*}
This reduces to
\[\left(\begin{array}{rrrrrr}
1 & 0 & -1 & 0 & 1 & 100 \\
0 & 1 & 1 & 0 & -1 & 100 \\
0 & 0 & 0 & 1 & 1 & 60 \\
0 & 0 & 0 & 0 & 0 & 0
\end{array}\right)\]
So $x_{1}, x_{2}, x_{4}$ are basic variables and $x_{3}, x_{5}$ are free variables. The solution set is
\[\left\{\begin{bmatrix} x_{3} - x_{5} + 100 \\ -x_{3} + x_{5} + 100 \\ x_{3} \\-x_{5} + 60 \\ x_{5}\end{bmatrix} | \forall x_{3}, x_{5} \in \mathbb{R}\right\}\]

If $x_{4} = 0$, then $x_{5} = 60$, so we get the solution set

\[\left\{\begin{bmatrix} x_{3} +40 \\ -x_{3} +   160 \\ x_{3} \\ 0  \\ 60\end{bmatrix} | \forall x_{3} \in [0, 160]\right\}\]

The minimum value of $x_{1}$ is $40$.
\subsection{1.7 -- Linear Independence}
\end{ex}
\begin{defn}
  Given a set of vectors $\vec{v}_{1}, \vec{v}_{2}, \vec{v}_{3}, \ldots$, we say that the set is \textbf{linearly independent}
  if and only if the only solution to
  \[c_{1}\vec{v}_{1} + c_{2}\vec{v}_{2} + c_{3}\vec{v}_{3} + \cdots = \vec{0} \qquad c_{i} \in \mathbb{R}\]
  is the trivial solution.

  We say the set is \textbf{linearly dependent} if and only if there is a solution to the equation
  \[c_{1}\vec{v}_{1} + c_{2}\vec{v}_{2} + c_{3}\vec{v}_{3} + \cdots = \vec{0}\qquad c_{i} \in \mathbb{R}\]
  which is not the trivial solution.
\end{defn}

\begin{thm}
  The columns of $\mathbb{A}$ are linearly independent if and only if the only solution to
  \[\mathbf{A}\vec{x} = \vec{0}\]
  is the trivial solution $\vec{x} = \vec{0}$.
\end{thm}
If one vector lies in the plane spanned by the other two, then the vectors are linearly dependent.
\begin{thm}
  To find out if a set of vectors is linearly dependent, RREF their matrix.
\end{thm}
\section{March 29}
No class due to doctor's appointment
\section{March 30}

\end{document}
