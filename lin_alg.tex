\documentclass[11pt]{article}
%  \usepackage{evan}
\usepackage{mathtools}
\usepackage{sagetex}
\usepackage{amsfonts}
\usepackage{amsmath}
%\usepackage{sectsty}
% \usepackage{graphicx}
\newtheorem{thm}{Theorem}
\newtheorem{defn}{Definition}
\newtheorem{ex}{Example}
% Margins
\topmargin=-0.45in
\evensidemargin=0in
\oddsidemargin=0in
\textwidth=6.5in
\textheight=9.0in
\headsep=0.25in

\title{Linear Algebra Notes}
\author{Atticus Kuhn}
\date{\today}


\begin{document}
\maketitle
\tableofcontents
\section{March 20}
\begin{defn}
  A \textbf{linear equation} of the variables $x_{1}, x_{2}, \ldots x_{n}$ is any equation of the form
  \[a_{1}x_{1}+a_{2}x_{2} + \cdots + a_{n}x_{n} = b\]
  where $a_{1}, \ldots a_{n}$ are called the coefficents and $b$ is called the constant.
  \end{defn}
  \begin{defn}
    A \textbf{system of linear equation} is a set of 1 or more linear equations.
    \end{defn}
    \begin{ex}
      \[x + 2y = 3\]
      is a linear system and

      \begin{align*}
        x-2y = 1 \\
        2x + 3y = 4 \\
        x + y =  5
        \end{align*}
        is a linear system.
        But
        \[5 \sqrt{x} - y=1\]
        is not a linear system.
      \end{ex}

      Given a linear system, we want to find all substitutions which satisfy the linear
      system.
      A \textbf{solution set} is a set of n-tuples, all of which are valid solutions to a given system of linear equations.
      Our central problem is: given a linear system, find its solution set.
      Big problems
      \begin{enumerate}
        \item Existance Question: Do there exist any solutions? If there are, we call the system \textbf{consistent}. If there are no solutions, we call the solution \textbf{inconsistent}.
              \item Uniqueness Question: How many solutions are there? If there is exactly 1 solution, we call that \textbf{unique}.
      \end{enumerate}
      \begin{thm}
        If a linear system has solutions, it either has 1 solution or infintely many solutions.
        A linear system cannot have 12 solutions.
      \end{thm}
      \begin{ex}
        \[x + 2y = 5, x + 4y = 6\]
        has solution
        \[x = -4, y=9/2\]
        This linear system corresponds to the matrix multiplication
        \[\begin{bmatrix} 1 & 2 \\ 3 & 4 \end{bmatrix} \begin{bmatrix} x \\ y \end{bmatrix} = \begin{bmatrix} 5 \\ 6 \end{bmatrix}\]
        We can turn this into an \textbf{augmented matrix} to get
        \[
          \begin{bmatrix}
            1 & 2 & 5 \\
            3 & 4 & 6
            \end{bmatrix}
        \]
        We are allowed to perform row operations on the rows to get
\[
          \begin{bmatrix}
            1 & 2 & 5 \\
            0 & 1 & 9/2
          \end{bmatrix}
          \implies
          \begin{bmatrix}
            1 & 0 & -4\\
            0 & 1 & 9/2
            \end{bmatrix}
        \]
        So $(x,y) = (-4, 9/2)$. The solution is unique.
      \end{ex}

      \begin{thm}
        We are allowed to do these elementary row operations
        \begin{defn}
          We say two systems of equations are \textbf{equivalent}
          if their solutions sets are equal.
          \end{defn}
        \begin{enumerate}
          \item replacement: replace one row with the sum of itself and another row
          \item interchange: swap 2 rows
                \item scaling: multiply all entries in a  row by a constant
                \end{enumerate}
        \end{thm}
        The reason the elementary operations work is because they
        give an equivalent system, i.e. they do not change the solution set.

        Convince yourself that each of the elementary operations do not change the solution
        set.
        \begin{ex}
          Let's solve this system
          \begin{align*}
            kx + y + 2z &=1 \\
            x + z &= h \\
            y-z &= 1
            \end{align*}
            The solution is
            \[x=-3h/(k-3), y=\frac{(h+1)k-3}{k-3}, z = \frac{hk}{k-3}\]
            The system would have no solutions (inconsistent) iff $k=3$ and $h \neq 0$.
            The system has infinite solutions if $k=3$ and $h= 0$.
            The system has 1 solution if $k \neq 3$.
          \end{ex}
          \begin{defn}
            A matrix is in \textbf{echelon form} iff
            \begin{enumerate}
              \item All nonzero rows are above any rows of all zeroes
              \item each leading entry in a row is in a column to the right of the leading
                    endtry of the row above it
                \item All entries in a column below a leading entry are zeroes
            \end{enumerate}
          \end{defn}
          \begin{defn}
            A matrix is in \textbf{reduce row echelon form} iff
            \begin{enumerate}
              \item The leading entry in each nonzero row is $1$
                    \item each leading $1$ is the only nonzero entry in its column
                    \end{enumerate}
            \end{defn}
            \begin{ex}
              This matrix
              \[
                \begin{bmatrix}
                  * & * & 0 & 0 & 0 & 0 \\
                  0 & * & 1 & 0 & 0 & 0 \\
                  0 & 0 & * & 0 & 0 & *
                  \end{bmatrix}
                \]
                is in REF, but not in RREF.
              \end{ex}
              \begin{ex}
                This matrix
                \[\begin{bmatrix}
                    0 & 1  & 0 & 0 & 0 \\
                    0 & 0 & 2 & 0 & 0 \\
                    0 & 0 & 0 & 0 & 1 \\
                    0 & 0 & 0 & 0 & 0
                    \end{bmatrix}\]
                  is in REF, and it would be in RREF if we changed the $2$ to a $1$.
                \end{ex}

                \begin{thm}
                A given matrix has infinitely many equivalent REF, but only 1 RREF.
                \end{thm}
                \begin{ex}
                  Reduce this matrix
                  \[
                    \begin{bmatrix}
                      1 & -1 & -1 \\
                      0 & 2 & 1 \\
                      2 & 0 & -1
                      \end{bmatrix}
                  \]
                  Solution:
                  \[
\left(\begin{array}{rrr}
1 & 0 & -\frac{1}{2} \\
0 & 1 & \frac{1}{2} \\
0 & 0 & 0
\end{array}\right)
                  \]
   \end{ex}
 \begin{defn}
   If a matrix is in REF, the \textbf{pivots}
   are leading entries. The \textbf{pivot columns} are
   the columns which contain a pivot.
\end{defn}
\begin{defn}
  A \textbf{basic variable} is a variable which corresponds to a pivot column.
  A \textbf{free variable} is a variable which does not correspond to a pivot column.
\end{defn}
\begin{ex}
  Given the matrix
  \[\begin{bmatrix}
      1 & 0 & 2 & 0 & 4 \\
      0 & 1 & 3 & 0 & 5 \\
      0 & 0 & 0 & 1 & 6 \\
    0 & 0 & 0 & 0 & 0 \end{bmatrix}\]
This matrix is in RREF. Variables $x_{1}, x_{2}, x_{4}$ are basic variables
and variable $x_{3}$ is a free variable.
\end{ex}

If the righmost column of $A$ is a pivot column, then we cannot tell if $\mathbf{A}\vec{x} = \vec{b}$ is consistent or inconsistent.
\begin{thm}
  If the rightmost column of
  $\left[\mathbf{A} \quad \vec{b}\right]$ is a pivot column, then
  $\mathbf{A}\vec{x} = \vec{b}$ is not consistent.
\end{thm}

\begin{thm}
  Even if you have a free variable, that doesn't guarantee infintely many solutions.

  If you have a free variable and the system is consistent, then the system has infinitely many solutions.
\end{thm}

\begin{ex}
  Given that the RREF is
  \[\begin{bmatrix}
      1 & 2 & 3 & 0 &  0 &  6 \\
      0 & 0 & 0  & 1 & 0 & 7 \\
      0 & 0 & 0 & 0 & 1 & 8\end{bmatrix}\]
  Describe the solution set.
  It is
  \[
    \begin{bmatrix}
      x_{1} \\ x_{2} \\ x_{3} \\ x_{4} \\ x_{5}
    \end{bmatrix}
    =
    x_{2} \begin{bmatrix} -2  \\ 1 \\ 0 \\ 0 \\ 0 \end{bmatrix}
    +
    x_{3} \begin{bmatrix} -3 \\ 0 \\ 1 \\ 0 \\ 0 \end{bmatrix}
    +
    \begin{bmatrix} 6 \\ 0 \\ 0 \\ 7 \\ 8 \end{bmatrix}
  \]
  This solution set is infinitely big because it has free varaibles.
  This is called \textbf{parametric vector form} of the solution set.
\end{ex}
\end{document}
