\documentclass[11pt]{article}
%  \usepackage{evan}
\usepackage{mathtools}
\usepackage{sagetex}
\usepackage{amsfonts}
\usepackage{amsmath}
%\usepackage{sectsty}
% \usepackage{graphicx}
\newtheorem{thm}{Theorem}
\newtheorem{defn}{Definition}
\newtheorem{ex}{Example}
% Margins
\topmargin=-0.45in
\evensidemargin=0in
\oddsidemargin=0in
\textwidth=6.5in
\textheight=9.0in
\headsep=0.25in

\title{Multi Notes}
\author{Atticus Kuhn}
\date{\today}


\begin{document}
\maketitle
\tableofcontents

% Optional TOC
% \tableofcontents
% \pagebreak

%--Paper--

\section{Jan 11. 2023}
\subsection{Syllabus}
Learning opportunity is a waste of time.
Teach your parents stuff, YHPL is not happy (it is meaningless).
Each knowledge check is worth more (20\%).
\subsection{Overview of 1D}
We are about to complete calculus.
You want to take real analysis or complex analysis after 1D.
\begin{table}[h]
  \centering
  \begin{tabular}{c|c}
    Class & Topic \\
    1A & single variable derivatives \\
    1b & single variable integrals \\
    1C & multi-variable derivatives \\
    1D & multi-variable integra;s \\
    \end{tabular}
  \end{table}



\subsubsection{Topics Covered in 1D}
Topics covered in 1D:
\begin{enumerate}
  \item work
\item line integral = work done by force
  \item vector fields
\item flux integral = rate of flow thru a thin net
  \item Chapter 20: Green's Theorem, curl, divergenence
    \item That's the end
\end{enumerate}
If we meet YHPL, she will bake us a cake.
\subsection{Review from 1C}
\subsubsection{review of quartic surfaces}
Review quadric surfaces
\begin{table}[h]
  \centering
  \begin{tabular}{||c c||}
    eqn & name \\
    $z = x^2 - y^ 2$ & Hyperbolic paraboloid \\
    $ z = y^2$ & Parabolic cylinder \\
    $z = \sqrt{4-x^2-y^2}$ & half-hemisphere of a sphere\\
    $z = \sqrt{x^2+y^2}$ & elleptic cone \\
    $ z = x^2 + y^2 $ & elliptic paraboloid \\
    $ z  =1 - \sqrt{x^2 + y^2 } $ & elliptic cone \\
    $ z = 6-3x-2y$ & plane \\
    $ z = 6-2y$ & plane thru $(37.0.6)$ and normal vector is $\vec{n} = \langle 0 ,2 ,1 \rangle$ \\
    $z = 4-x^2-y^2$ &  Elliptic Paraboloid\\
   \end{tabular}
  \end{table}
Review table that YHPL posted on quartic surfaces.
\subsubsection{review of integrals}
Recall that the definition of integral is
\[\int_I f(x) \, dx = \int_a^b f(x) \, dx = \lim_{n \to \infty} \sum_{i=1}^n f(x_i^*) partialta x_i\]
where $f(x_i^*)$ is the height of the $i^{th}$ rectangle and $partialta x_i$ is the width of the
$i^{th}$ rectangle. We are summing up areas on many small rectangles. Note that area can be
negative if the function goes below the x-axis (called signed area).
Note that all the $partialta x$s do not all have to be the same width.

\begin{thm}
  Average
$avg = \frac{1}{|I|} \int_I f(x) \, dx$
\end{thm}
\subsection{Multi-Variable integration}
Given that $z = f(x,y)$, we find the area by splitting the region into rectangles under the
curve. We split the x-axis and we split the y-axis. We integrate over a region $R$.
Adding up the volume of all the little rectangular prisms approximates the volume of
our original curve.

\begin{defn}
  The definition of a \textbf{multi-variable} integral is
  \[\iint_R f(x,y) \,dx \,dy = \int_R f(x,y) dA = \lim_{n\to \infty} \lim_{n \to \infty} \sum_{i=1}^n \sum_{j=1}^m f(x_i^*, y_j^*) partialta x_i partialta y_j\]
\end{defn}
In a double integral, you have to integral twice.
\subsubsection{numerical approximation in multiple variables}
Let's use this table as an example of a function

\begin{table}
  \centering
  \begin{tabular}{c c c c c}
    x y &1 &4& 7 &10\\
    4 & 5 & 6 & 9 & 37 \\
    2 & 11 & 14 & 7 & 12 \\
    0 & 16 & 21 & 0 & 15
  \end{tabular}
\end{table}
Let $R = [1,10] \times [[0,4]]$.
\[\int_R f dA \approx (14 + 6 + 9 + 37 + 12 + 7)\cdot3\cdot 2\]

Example: Let $z = e^{-(x^2+y^2)}$. If you sample the function using the bottom left approximation,
then this is an overestimate because the rest of the rectangle with have higher $z$-values.

Given a contour map, we can approximate an integral. Divide up the graph and pick a point
from each division to represent the whole.
\subsection{Summary of surfaces}
\begin{sageblock}
var('x y z')
\end{sageblock}
\begin{table}[h]
  \centering
  \begin{tabular}{c c c }
    Name & Equation & Graph \\
    Ellipsoid  & $\frac{x^2}{a^2} + \frac{y^2}{b^2} + \frac{z^2}{c^22} = 1$ &  \sageplot[height=3cm]{implicit_plot3d(x^2/2+y^2/3  + z^2/4 == 1 , (x,-2, 2), (y, -2, 2), (z, -2, 2)),frame=False} \\
    Cone & $\frac{x^2}{a^2} + \frac{y^2}{b^2} = \frac{z^2}{c^2}$ &   \sageplot[height=3cm]{implicit_plot3d(x^2/2+y^2/3 == z^2/4 , (x,-2, 2), (y, -2, 2), (z, -2, 2),frame=False)} \\
    Elliptic Paraboloid & $\frac{x^2}{a^2} + \frac{y^2}{b^2} = \frac{z}{c}$&   \sageplot[height=3cm]{implicit_plot3d(x^2/2+y^2/3 == z/4 , (x,-2, 2), (y, -2, 2), (z, -2, 2),frame=False)} \\
    Hyperboloid of One Sheet & $\frac{x^2}{a^2} + \frac{y^2}{b^2} - \frac{z^2}{c^2} = 1$ &   \sageplot[height=3cm]{implicit_plot3d(x^2/2+y^2/3 - z^2/4==1 , (x,-2, 2), (y, -2, 2), (z, -2, 2),frame=False)} \\
    Hyperboloid of Two Sheets & $\frac{x^2}{a^2} - \frac{y^2}{b^2} = \frac{z}{c}$ &   \sageplot[height=3cm]{implicit_plot3d(x^2/2+y^2/3 == z/4 , (x,-2, 2), (y, -2, 2), (z, -2, 2),frame=False)} \\
    Hyperbolic Paraboloid & $-\frac{x^2}{a^2} - \frac{y^2}{b^2} + \frac{z^2}{c^2}$=1 &   \sageplot[height=3cm]{implicit_plot3d(-x^2/2-y^2/3 + z^2/4==1 , (x,-2, 2), (y, -2, 2), (z, -2, 2),frame=False)} \\
    \end{tabular}
  \end{table}
\section{January 12, 2023}
no class due to doctor's appointment
\section{January 15, 2023}
No class due to MLK day
\section{January 18, 2023}
If we are integrating over a non-rectangular region, the inner bounds must depend on the outer bounds.
\subsection{Integrating non-rectangular regions}
Example:

A strange region may be given by a triangle an semi-cirle

\[\int_{x=-1}^{x=0} \int_{y=1}^{y=x+2} f dydx + \int_{x=0}^{x=\sqrt{3}} \int_{y=2}^{y=\sqrt{4-x^2}} f dy dx\]
But we could also do as y

\[\int_{y=1}^{y=2} \int_{x=-1}^{x=y-2} f dx dy + \int_{y=1}^{y=2} \int_{x=0}^{x=\sqrt{4-y^2}} f dx dy\]
Note that it could be (no need to split)
\[\int_{x=1}^{x=2} \int_{y=1}^{y=\sqrt{4-x^2}} f dx dy\]
Excercise:
Given the integral
\[\int_{-1}^0 \int_1^{4-x} f dy dx\]
reverse the order of the integrals
\[\int_{y=1}^{y=4} \int_{x=-1}^{x=0} f dx dy  + \int_{y=4}^{y=5} \int_{x=-1}^{x=4-y} f dx dy\]

\subsection{Application of Double Integrals}
We use double integrals to find volume.
\begin{ex}
Exercise: given this weird parabola, find volume.
\sageplot[height=5cm]{implicit_plot3d(z == 25 - x^2 - y^2, (x, -3, 3), (y, -3, 3), (z, 16, 25))}
\[s_1: z= 25 - x^2 - y^2, s_2: z = 16\]

To find volume
\[\int_{x=-3}^{x=3} \int_{y=-\sqrt{9-x^2}}^{y=\sqrt{9-x^2}} 9 - x^2-y^2 dy dx\]
Pro =-Tip: always project orthoganlly onto the xy plane
\end{ex}

Another example
\begin{ex}
Given the mass-density of
\[s(x,y) = \sqrt{x^2+y^2}\]
the mass of a triangle is given by
\[\int_{x=0}^{x=2} \int_{y=0}^{y=4-2x} s(x,y) dy dx\]
\end{ex}
\begin{ex}
Example:
Given a city with the shape of a semi-circle. Find the average distance
from the city to the ocean.

YHPL strongly recommends reading ahead.
\sageplot[height=5cm]{plot(sqrt(9-x^2), (x, -3, 3)) + plot(0, (x, -3, 3))}

The average distance from the city to the ocean is given by
\[\frac{\int_{x=-3}^{x=3} \int_{y=0}^{y=\sqrt{9-x^2}} y \, dy dx}{\int_{x=-3}^{x=3} \int_{y=0}^{y=\sqrt{9-x^2}} 1 \, dy dx}\]

To get the average distance to the ocean, we get the total distance and
then divide by the total area.
\end{ex}

\begin{ex}
Without computation, find the sign of
\[\int_R (y^3-y) dA\]
and
\[\int_R e^yx dA\]
where
the region $R$
is
\sageplot[height=5cm]{plot(-sqrt(1-x^2), (x, -1, 1))}?

Solution:
Remember the definition. The first integral will be negative because
$y < 0 \implies y^3 > y \implies y^3 - y > 0$.
The second integral will be 0, because the negative $x$ perfectly cancels out the positive $x$.

\end{ex}

\begin{ex}
 What is the sign of
 \[\int_R cos(x) dA\]
 where $R$ is the same region as above?
 Solution: It is positive because $\cos{x} > 0$ for $-1 < x < 1$. Cosine only becomes 0 at $\pi/2 \approx 1.5$.
\end{ex}


\subsection{Triple Integral}
\begin{defn}
  Given a funciton $w = f(x,y,z)$ and a region $W \subset \mathbb{R}^3$
  \[\iiint_w f(x,y,z) dV\]
\end{defn}
There are $3! = 6$ different ways to integrate a triple integral. Where the inner integrals can depend on
the outer integrals.

\begin{ex}
  Redo the parabola volume question using a triple integral.
  \[\int_{x=-3}^{x=3} \int_{y=-\sqrt{9-x^2}}^{y=\sqrt{9-x^2}} \int_{z=16}^{z=25-x^2-y^2} 1 \, dz dy dx\]
\end{ex}


\section{January 19, 2023}
\subsection{Review}
Review of what we learned last week
\begin{enumerate}
  \item find volume by integrating over 1
  \item determine sign of integral by using sign of integrand over region
  \item You can swap the order of integration (either $dx dy$ or $dy dx$), and sometimes
    one order of integration will be easier than the other
  \item When approximating a double integral, you pick a point from each region to approximate the
    entire region. From a contour map, you can choose the sample point and then multiply that
    by the area of the region.
  \item you can find average function value by taking double integral over region and
    then divide by the area of that region.
\end{enumerate}
\begin{ex}
  Here's an application problem:
  Estimate the average snowfall in Colorado based on this map. Sample based on the midpoint of each rectangular region.
  \[\frac{1}{16}(16+16+19+13+8+28+18+13+2+24+17+11+0+16+8+7) = \frac{27}{2}\]
\end{ex}
\subsection{16.4 -- Polar Coordinates}
In polar coordinates, $(r, \theta)$, a point is represented by its distance from the origin, $r$, and the angle it makes with the positive $x$-axis, $\theta$.


\begin{ex}
Let's revisit the problem of the average distance from the city to the ocean that we did
last week.
The same integral becomes
\begin{align*}
\int_{\theta = 0}^{\theta = \pi} \int_{r=0}^{r=3} r^2 \sin(\theta) dr d\theta \\
&= \int_0^\pi \sin \theta d\theta \cdot \int_0^3 r^2 dr\\
&= (\cos \pi - \cos 0)(\frac{27}{3})\\
&= -18
\end{align*}
\end{ex}

\begin{ex}
  Another example: We plot over the region
  \sageplot[height=5cm]{polygon([[0,0], [0, 2], [1,2]], color='red')}
  \begin{align*}
    \int_{x=0}^{x=1}\int_{y=2x}^{y=2} x dy dx \\
    &= \int_{\theta = \arctan 2}^{\theta = \pi/2} \int_{r=0}^{r=2/\sin \theta} r^2 \cos \theta dr d\theta\\
  \end{align*}

  If you want $r$ to be the outer, then it's a bit harder
  We split it into 2 regions
  \sageplot[height=5cm]{polygon([[0,0], [0, 2], [1,2]], color='red') + plot(sqrt(4-x^2), (x, 0, 2))}
  \begin{align*}
    \int_{r=0}^{r=2} \int_{\theta = \arctan 2}^{\theta = \pi/2} r^2 \cos \theta d\theta dr
    + \int_{r=2}^{r=\sqrt{5}} \int_{\theta = \arctan 2}^{\theta = \arcsin 2/r} r^2 \cos \theta d\theta dr
  \end{align*}
\end{ex}
\subsubsection{Polar--Rectangular Conversions}
\[x = r\cos \theta\]
\[y=r\sin\theta\]
\[x^2+y^2=r^2\]
\[\theta = \arctan \frac{y}{x}\]
\[partialta A \approx r partialta \theta partialta r\]
\[dA = r\, d\theta dr = r \, dr d\theta\]
\[\int_R f dA = \int_\alpha^\beta\int_a^b f(r\cos\theta, r\sin \theta) drd\theta\]
\section{Chapter 16 Knowledge Check Practice}
\begin{enumerate}
  \item  Consider the double integral $\int_R partialta(x,y) dA$ where $partialta(x,y)$ is the distance from $(x,y)$ to $(0,0)$ and $R$ is the region bounded by the y-axis, the line $y=1$, the line $x=1$ and the semi-cirle $y=\sqrt{5-x^2}$
    \renewcommand{\theenumi}{\Alph{enumi}}
  \begin{enumerate}
    \item Draw $R$\\
      Solution: \sageplot[height=5cm]{region_plot(y<=sqrt(5-x^2), (x, 0, 1), (y, 1,3))}
    \item find $partialta(x,y)$.\\
      Solution: $partialta(x,y) = \sqrt{x^2+y^2}$
    \item What is the practical meaning of the double integral\\
      Solution: the total mass of the region $R$.
    \item Write the double integral of the form $dxdy$. Do not evaluate.\\
      Solution: \[\int_{y=1}^{y=2} \int_{x=0}^{x=1} \sqrt{x^2+y^2} \, dx dy + \int_{y=2}^{y=\sqrt{5}} \int_{x=0}^{x=\sqrt{5-y^2}} \sqrt{x^2+y^2} \, dx dy\]
    \item Write the double integral of the form $dydx$. Do not evaluate.\\
      Solution: \[\int_{x=0}^{x=1} \int_{y=1}^{y=\sqrt{5-x^2}} \sqrt{x^2+y^2} \, dy dx\]
    \item Write the double integral of the form $drd\theta$. Do not evaluate.\\
      Solution: \[\int_{\theta=\pi/4}^{\theta=\arctan(2)} \int_{r=\sec\theta}^{r=\csc\theta} r^2 dr d\theta + \int_{\theta=\arctan 2}^{\theta=\pi/2} \int_{r=\sec\theta}^{r=\sqrt{5}} r^2 dr d\theta\]
      \item Write the double integral; of the form $d\theta dr$. Do not evaluate.\\
        Solution: \[\int_{r=1}^{r=\sqrt{2}} \int_{\theta=\arccos(1/r)}^{\theta=\pi/2} r^2 d\theta dr + \int_{r=\sqrt{2}}^{r=\sqrt{5}} \int_{\theta=\arcsin(1/r)}^{\theta=\pi/2} r^2 d\theta dr\]
      \item Set up an integral to express the average mass density of $r$\\
        Solution: \[\frac{\int_{x=0}^{x=1} \int_{y=1}^{y=\sqrt{5-x^2}} \sqrt{x^2+y^2} \, dy dx}{\int_{x=0}^{x=1} \int_{y=1}^{y=\sqrt{5-x^2}} 1 \, dy dx}\]
  \end{enumerate}
  \item A solid region $W$ is bounded above by $z=5-x^2-y^2$ and below by $z=1$.
    \begin{enumerate}
      \item Sketch $W$ in $R^3$, and the projection of $W$ onto the $xy$-plane  \\
        Solution: \sageplot[height=5cm]{implicit_plot3d(z==5-x^2-y^2, (x,-2,2 ), (y, -2,2), (z, 1, 5), region=(lambda x,y,z: z >=1))}
        and
        \sageplot[height=5cm]{implicit_plot(x^2+y^2==4, (x, -2, 2), (y, -2, 2))}
    \end{enumerate}
  \item Reverse the order of the integral
    \[\int_{x=-1}^{x=0} \int_{y=1}^{y=4-x} f(x,y) dy dx\]\\
    Solution:
    The first step is always to draw a picture of the region.
    \sageplot[height=5cm]{region_plot(y<=4-x, (x, -1, 0), (y, 1, 5))}
    \[\int_{y=1}^{y=4} \int_{x=-1}^{x=0} f(x,y) dx dy + \int_{y=4}^{y=5} \int_{x=-1}^{x=4-y} f(x,y) dx dy\]
\end{enumerate}
\section{January 23, 2023}
\begin{ex}
  Consider the region
  \sageplot[height=7cm]{region_plot(y<=4, (x, -1, 2), (y, -3, 4))}
  The limits of the integral
  \[\int_{x=-1}^{x=2} \int_{y=-3}^{y=4} f dy dx\]
  are constant because we are using CARTESIAN coordinates.
  But, by contrast, consider
  a semicircle. This will have constant limits of integration in POLAR coordaintes
\end{ex}
\begin{ex}
  Consider the region
  \sageplot[height=7cm]{region_plot(y<=2, (x, -1, 0), (y, 0, 2)) + region_plot(y<=sqrt(4-x^2), (x, 0, 2), (y, 0,2))}
  Do 4 different orders of integration
  First, $dydx$
  \[\int_{x=-1}^{x=0} \int_{y=0}^{y=2}x  \, dydx + \int_{x=0}^{x=2} \int_{y=0}^{y=\sqrt{4-x^2}} x dy dx\]
  Next $dx dy$
  \[\int_{y=0}^{y=2} \int_{x=-1}^{x=\sqrt{4-y^2}} x \,dx dy\]
  Next $dr d\theta$
  \[\int_{\theta = 0}^{\theta = \pi/2} \int_{r=0}^{r=2} r^2 \cos \theta dr d\theta +
  \int_{\theta = \pi/2}^{\theta = \pi/2 + \arctan(1/2)} \int_{r=0}^{r=2\csc \theta} r^2\cos\theta dr d\theta +
  \int_{\theta = \pi/2 + \arctan(1/2)}^{\theta = \pi} \int_{r=0}^{r=-\sec\theta} r^2\cos \theta dr d\theta \]
  Finally $d\theta dr$
  \[
  \int_{r=0}^{r=1} \int_{\theta = 0}^{\theta = \pi} r^2 \cos \theta d\theta dr
  + \int_{r=1}^{r=2} \int_{\theta=0}^{\theta = \arccos(-1/r)} r^2\cos\theta d\theta dr
  + \int_{r=2}^{r=\sqrt{5}} \int_{\theta=\arcsin(2/r)}^{\theta=\arccos(-1/r)} r^2\cos\theta d\theta dr.\]
\end{ex}
By the way, our first knowledge check on chapter 16  is on 01/26
\begin{ex}
  Imagine a circular dinner plate with radius 10cm where the mass density of the dinner plate is given by
  \[partialta(x,y) = \sqrt{x^2+y^2}\]
  use polar coordinates to find the total mass of the plate.\\
  Solution:
\begin{align*}
  \int_{\theta  = 0 } ^ {\theta = 2\pi} \int_{r=0}^{r=10} r^2 dr d\theta \\
  &= \int_{\theta = 0}^{\theta = 2\pi} 1000/3 d\theta \\
  &= 2\pi*1000/3 = \frac{2000\pi}{3}
\end{align*}
\end{ex}
\begin{ex}
  What is
  \[\int_{-\infty}^{\infty} e^{-x^2} dx?\]
  Solution:
  Let
  \[I = \int_{-\infty}^{\infty} e^{-x^2}dx\]
  Then
\begin{align*}
  I^2 &= \int_{-\infty}^{\infty} e^{-x^2}dx\int_{-\infty}^{\infty} e^{-y^2}dy \\
  &= \int_{x=-\infty}^{x=\infty} \int_{y=-\infty}^{y=\infty} e^{-(x^2+y^2)} dy dx \\
  &= \int_{\theta = 0}^{\theta = 2\pi} \int_{r=0}^{r=\infty} e^{-r^2}r dr d\theta\\
  &= \pi
  \end{align*}
So $I=\sqrt{\pi}$
\end{ex}
\subsection{16.5 - Cylindrical and Spherical Coordinates}

\subsubsection{Cylindrical Coordinates}
Cylindrical coordinates are like polar coordinates in 3D. It uses 1 angle and 2 lengths.
They are described by $(r, \theta, z)$
The conversion is
Rectangular to cylindrical:
\[(x,y,z) \to (\sqrt{x^2+y^2}, \arctan(y/x), z)\]
cylindrical to rectangular:
\[(r,\theta, z) \to (r\cos \theta, r\sin \theta, z)\]
The reason they are called cylindrical coordiantes is because if you are trying to integrate a cylinder, the bounds of integration
are constant
\subsubsection{Spherical Coordaintes}
Spherical Coordinates use 2 angles and 1 length. The coordinates are of the form $(\rho, \theta, \phi)$
The conversions are
\begin{align*}
  \rho^2 &= x^2+y^2+z^2\\
  x &= \rho\sin\phi\cos\theta\\
  y &=\rho\sin\phi\sin\theta\\
  z &= \rho\cos\phi\\
  r &=  \rho\sin\phi\\
  \phi &= \arccos(\frac{z}{\sqrt{x^2+y^2+z^2}}) \\
  dV &= \rho^2 \sin\phi d\rho d\phi d\theta
\end{align*}
\section{January 25, 2023}
\begin{ex}
  Find the volume of this cone using all integration methods.
  \sageplot[height=10cm]{implicit_plot3d(z^2 == x^2+y^2, (x, -2, 2), (y, -2, 2), (z, 0, 2))}\\
  Solution:
  Note that the equation for this cone is $x^2+y^2 = z^2$
  \[\int_{x=-2}^{x=2}, \int_{y=-\sqrt{4-x^2}}^{y=\sqrt{4-x^2}} 2-\sqrt{x^2+y^2} dy dx\]
  \[\int_{\theta=0}^{\theta = 2\pi} \int_{r=0}^{r=2} r(2-r) dr d\theta\]
  \[\int_{\theta = 0}^{\theta = 2\pi} \int_{r=0}^{r=2}\int_{z=r}^{z=2} r dz dr d\theta\]
  \[\int_{x=-2}^{x=2} \int_{y=-\sqrt{4-x^2}}^{\sqrt{4-x^2}} \int_{z=\sqrt{x^2+y^2}}^{z=2} 1 dz dy dx\]
  \[\int_{\theta = 0}^{\theta = 2\pi} \int_{\phi =0 }^{\phi = \pi/4} \int_{\rho = 0}^{\rho = 2\csc\phi} \rho^2 \sin\phi d\rho d\phi d\theta\]

\end{ex}

\begin{ex}
  Consider the 3d solid which is a dome on the bottom and  a
  cylinder on the top.
  \sageplot[height=10cm]{
    plot3d(x^2+y^2, (x,-1, 1), (y, -1, 1))
    + implicit_plot3d(x^2 + y^2 ==1, (x, -1,1), (y, -1,1), (z, 1, 2))
    + plot3d(2, (x, -1,1), (y, -1, 1))}
  Find the mass given that the mass density for a point is given by the
  distance from $z=2$.
  \[\int_{\theta = 0}^{\theta = 2\pi} \int_{r=0}^{r=2} \int_{z=r }^{z = 2} (2-z)r dz dr d\theta\]
\begin{align*}
  &\int_{\theta = 0}^{\theta = 2\pi} \int_{\phi = 0} ^{\phi = \arctan 1/2} \int_{\rho = 0}^{\rho = 2\sec\phi} \rho^2 \sin\phi d\rho d\phi d\theta\\
  &+\int_{\theta = 0}^{\theta = 2\pi} \int_{\phi = 2\csc\theta} ^{\phi = \pi/2} \int_{\rho =0 }^{\rho = \csc\phi}(2-\rho\sin\phi) \rho^2 \sin\phi d\rho d\phi d\theta\\
  &+\int_{\theta = 0}^{\theta = 2\pi} \int_{\phi = \arctan 1} ^{\phi = \pi/2} \int_{\rho = 0}^{\rho = \cos\phi/\sin^2\phi} (2-\rho\sin\phi)\rho^2 \sin\phi d\rho d\phi d\theta\\
\end{align*}
  \[\int_{x=-1}^{x=1} \int_{y= -\sqrt{4-x^2}}^{y=\sqrt{4-x^2}} \int_{z=x^2+y^2}^{z=2} 2-z dz dy dx\]
  The spherical integral seems complicated, so you should break up the shape.
  \sageplot[height=7cm]{region_plot(y >= x^2, (x, -1, 1), (y, 0, 1)) + region_plot(y<=2, (x, -1, 1), (y, 1, 2))}
\end{ex}
\begin{ex}
  Write an integral for $f$
  for the region
  \sageplot[height=8cm]{plot3d(1-sqrt(x^2+y^2), (x, -1, 1), (y, -1, 1)) + plot3d(-1+x^2+y^2, (x, -1, 1), (y, -1, 1))}
  which is bounded by $z = 1-\sqrt{x^2+y^2}$ and $z = -1+x^2+y^2$
  \[\int_{\theta = 0}^{\theta = 2\pi} \int_{r=0}^{r=1} \int_{z= r^2-1}^{1-r} fr \, dz dr d\theta \]
  \[\int_{x=-1}^{x=1} \int_{y=-\sqrt{1-x^2}}^{y=\sqrt{1-x^2}} \int_{z=-1+x^2+y^2}^{1-\sqrt{x^2+y^2}} f\, dz dy dx\]
  Remember that $z=1-\sqrt{x^2+y^2} \implies \rho = \frac{1}{\cos\phi + \sin\phi}$
  \[\int_{\theta = 0}^{\theta = 2\pi} \int_{\phi = 3\pi/4}^{\phi=\pi} \int_{\rho = 0}^{\rho = }\]
\end{ex}
\section{Knowledge Check Solutions}
Consider the double integral
\[\int_{1}^{2} \int_{1}^{\sqrt{9-x^{2}}} partialta(x,y) dA\]
where $partialta(x,y)$ is the distance from the y-axis
\begin{enumerate}
  \item Sketch $R$ \\
        \sageplot[height=7cm]{region_plot([y<sqrt(9-x^2), y > 1], (x, 1, 2), (y, 1, 3))}
  \item find $partialta(x,y)$ \\
        Solution: $partialta(x,y) = x$
  \item What is the practical meaning of the integral?\\
        Solution: The total mass
  \item dydx
        \[\int_{1}^{2}\int_{1}^{\sqrt{9-x^{2}}} x dy dx\]
  \item dx dy
    \[\int_{1}^{\sqrt{5}} \int_{1}^{2} x dx dy + \int_{\sqrt{5}}^{\sqrt{8}} \int_{1}^{\sqrt(9-y^{2})} x dx dy \]
  \item $dr d\theta$
        \[\int_{\arctan(1/2)}^{\arctan(1/1)} \int_{1/\sin\theta}^{2/\cos\theta} r^{2}\cos\theta dr d\theta + \int_{\arctan(1)}^{\arctan(\sqrt{5}/2)} \int_{1/\cos\theta}^{2/\cos\theta} r^{2}\cos\theta dr d\theta
        + \int_{\arctan(\sqrt{5}/2)}^{\arctan(\sqrt{5})} \int_{1/\cos\theta}^{3} r^{2}\cos\theta drd\theta\]

  \item $d\theta dr$
        \[\int_{\sqrt{2}}^{\sqrt{5}} \int_{{\arcsin(1/r)}}^{\arccos(1/r)} r^{2}\cos\theta d\theta dr + \int_{\sqrt{5}}^{3} \int_{\arccos(2/r)}^{\arccos(1/r)} r^{2} \cos\theta d\theta dr\]
  \item Let $W$ be the solid defined by $z \le 8-x^{2}-y^{2}$, $x^{2 } + y^{2} \le 4, z \ge -3$. Sketch $W$.
  \item dydx \\
        Solution: \[\int_{-2}^{2} \int_{\sqrt{4-x^{2}}}^{\sqrt{4-x^{2}}} (11-x^{2}-y^{2})\]
  \item $drd\theta$ \\
        Solution: \[\int_{0}^{2\pi} \int_{0}^{2} r(11-r^{2}) drd\theta\]
  \item dzdydx \\
        Solution: \[\int_{-2}^{2}\int_{-\sqrt{4-x^{2}}}^{\sqrt{4-x^{2}}} \int_{-3}^{8-x^{2}-y^{2}} 1 dz dy dx\]
  \item $dzdrd\theta$ \\
        Solution: \[ \int_{0}^{2\pi} \int_{0}^{2} \int_{-3}^{8-r^{2}} r dzdrd\theta\]
  \item $d\rho d\phi d\theta$ \\
        Solution :
\begin{align*}
        &\int_{0}^{2\pi} \int_{0}^{\arctan(2/4)} \int_{0}^{\frac{-\cos\phi + \sqrt{\cos^{2}  \phi + 32\sin^{2}}}{2\sin^{2} \phi}} \rho^{2} \sin\phi d\rho d\phi d\theta\\
        &+ \int_{0}^{2\pi} \int_{\arctan(2/4)}^{\pi - \arctan(2/3)} \int_{{0}}^{2/\sin\phi} \rho^{2} \sin\phi d\rho d\phi d\theta\\
  &+ \int_{0}^{2\pi} + \int_{\pi-\arctan(2/3)}^{\pi} \int_{0}^{-3/\cos\phi} \rho^{2} \sin\phi d\rho d\phi d\theta\\
    \end{align*}
  \item Use cylindrical coordiantes to compute the total mass if the mass density is the distance from the origin \\
        Solution \[\int_{0}^{2\pi} \int_{0}^{2} \int_{-3}^{8-x^{2}-y^{2}} r (r^{2} + z^{2}) dz dr d\theta\]
\end{enumerate}
\section{January 30, 2023 - Vector Fields}
\begin{ex}
Remember  that you can parameterize the graph of a circle by
\[\vec{r}(t) = \langle a\cos t, a \sin t \rangle , 0 \le t \le 2\pi`\]
If we add the angular frequency $\omega$, we get a more generalized form
\[\vec{r}(t) = \langle a\cos \omega t, a \sin \omega t \rangle , 0 \le t \le \frac{2\pi}{\omega}`\]
And the speed of this particle is $a \omega$

Remeber that $\vec{v} = \frac{d\vec{r}}{dt}$,
so
\[\vec{v}(t) = \langle -a\omega \sin(\omega t), - a \omega \cos (\omega t) \rangle\]
so $\|\vec{v}\| = a \omega$

Let's take the derivative again to get
\[\vec{a}(t)  = \langle -a \omega^{2} \cos(\omega t), -a \omega^{2} \sin(\omega t) \rangle\]

Note that $\vec{r}$ is always perpendicular to $\vec{v}$ because
$\vec{r} \cdot \vec{v} = 0$. Also note that $\vec{a}  = -\omega^{2} \vec{r}$.

Don't overgeneralize: $\vec{v}$ is not always perpendicular to $\vec{r}$, but it is perpendicular in circular motion.
\end{ex}
\begin{ex}
  Consider the following graph
  \sageplot[height=6cm]{parametric_plot((2*cos(x), 2*sin(x)), (x, 0, pi))}
  Which is traced out by a particle over 5 seconds. Find the location of the particle at time $t$.\\
  Solution: The equation of the particle is
  \[\vec{r}(t) = \langle 2 \cos(-\pi t/5), 2 \sin (-\pi t/5) \rangle\]
  You can also paramtetrize $(r, \theta)$ by $t$.
  \[r = 2, \qquad\theta = \pi - \pi t/5\]
\end{ex}

\begin{ex}
  A particle moves in a circle of radius $2$ at $5 m/s$. Find the equation of the particle. \\
  Solution:
  \[\vec{r} = \langle 2 \sin (5 t /2) , 2 -\cos (5t / 2)\rangle,  0 \le t \le \pi / 5 \]

\end{ex}
\begin{defn}
  A \textbf{vector field} is a function from $\mathbb{R}^{n} \to \mathbb{R}^{n}$. It takes in a vector and spits out a vector.  For example,
  \[F(x,y) = \langle x + y, x - y \rangle\]
\end{defn}
From now on, everything is about vector fields.

The way to sketch a vector field in $\mathbb{R}^{2}$ is to draw a little arrow at each point representing the output vector.
\begin{ex}
  For example, here is a sketch of
$F(x,y) = \langle x + y, 2x - y \rangle$\\
\sageplot[height=5cm]{plot_vector_field((x + y, 2*x  - y), (x, -3, 3), (y, -3, 3))}
\end{ex}

\begin{ex}
  sketch $F(x,y) = \langle x, y \rangle$ and $F(x,y)  = \langle -y, x \rangle$?
  Solution:\\
\sageplot[height=5cm]{plot_vector_field((x,y), (x, -3, 3), (y, -3, 3))}\\
\sageplot[height=5cm]{plot_vector_field((-y, x), (x, -3, 3), (y, -3, 3))}\\
\end{ex}
\section{Februrary 1 2023}
\subsection{17.3 - Flow Line}
\begin{defn}
  The \textbf{flow line} is
  how a particle in a vector field would ``flow'' (imagine the vector field is a force exerted on the particle).
  A path $\sigma(t) : \mathbb{R} \to \mathbb{R}^{n}$ is a flow-line in a vector field $\vec{F}(x) : \mathbb{R}^{n} \to \mathbb{R}^{n} $ iff
  \[\sigma'(t) = \vec{F}(\sigma(t))\]
\end{defn}

Let's review parametric functions
\begin{ex}
  Find the flow line defined by $\vec{r}(t)$
  if the vector field is $\vec{F} = \langle 1,3 \rangle$ and  $\vec{r}(0) = \langle 2, 4 \rangle$\\
  Solution:
  \[\vec{r}(t) = \langle 2 + t, 4 + 3t \rangle \]
\end{ex}
\begin{ex}
  Find the flow line of $\vec{r}(t)$ if
  the vector field is $\vec{F}(x,y) = \langle y, 2y \rangle$
  and $\vec{r}(1) = \langle 3,4 \rangle$\\
  Solution:
  \begin{align*}
    r' &= F \circ r \\
    \langle x', y' \rangle &= \langle y, 2y \rangle \\
\vec{r}(t) &= \langle 2 \, e^{\left(2 \, t\right)} + 1, 4 \, e^{\left(2 \, t\right)}
\rangle
  \end{align*}
\end{ex}
Note that you can find flow lines in sage using
\begin{sageblock}
# I am going to do this later
print('hello world')
\end{sageblock}

\begin{ex}
  Find the flowline if the field is $\vec{F}(x,y) = \langle 2y, 1 \rangle$
  and $\vec{r}(0) = \langle 3, 4 \rangle$ \\
  Solution:\\
  Note that a sketch can help you find the solution\\
\sageplot[height=5cm]{plot_vector_field((2*y, 1), (x, -3, 3), (y, -3, 3))}\\
  \begin{align*}
    x' = 2y &, y' = 1, x(0) = 3, y(0) = 4 \\
    y &= t +  4\\
    x &= t^{2}  + 8t +  3\\
    \vec{r}(t) &= \langle t^{2} + 3, t  + 4\rangle
  \end{align*}
\end{ex}
\begin{ex}
  Consider the vector field $\vec{F}(x,y) = \langle y, x \rangle$ and $\vec{r}(t) = \langle 3, 4 \rangle$
  Solution: \\
  \[x\left(t\right) = -\frac{1}{2} \, e^{\left(-t\right)} + \frac{7}{2} \, e^{t}, y\left(t\right) = \frac{1}{2} \, e^{\left(-t\right)} + \frac{7}{2} \, e^{t}
\]
\end{ex}
\subsection{17.4 -- Euler's Method}
If we cannot find an exact solution, we can do a numerical approximation using Euler's Method (approximate differential equations with a tangent line).
The main idea is that
\[f(a + partialta x) \approx f(a) + partialta x f'(a)\]
\section{Chapter 17 Practice Knowledge Check}
\begin{enumerate}
  \item A particle moves in the direciton of $\langle 4,3 \rangle $ along a straight line at a constant speed of 10 and is located at $(1,2)$ and $(a, 0)$ when $t=k$
        and $t=5$ respectively. Find $\vec{r}$
        Solution: \[\vec{r}(t) = \langle -31 + 8t, -22 + 6t \rangle\]
  \item A particle is located at $(0, -4)$ and moves along a full circlular path clockwise centered at the origin at a cosntant
        speed of $\pi$. Find the particle's veclocity at $t=2$\\
        Solution: \[4\pi\]
        (note that this question was dumb and $\pi$ referred to )
  \item let $\vec{F}(x,y) = \langle 2, 4y \rangle$ be ocean current. An ice berg is at $(3,5)$ at $t=0$.
        \begin{enumerate}
          \item Use Euler's Method with two steps to approximate the location of the iceberg at t=1. \\
                Solution:
                \begin{align*}
                    r(0) &= \langle 3, 5\rangle \\
                  r(1/2) &\approx \langle 3, r \rangle + 1/2 \rangle 2, 4 * 5 \rangle = \langle 4, 15 \rangle \\
                  r(1) &\approx \langle 4, 15 \rangle + 1/2 \langle 2, 4*15 \rangle = \langle \rangle
                  \end{align*}
          \item Find the exact location of the iceberg in the ocean current at $t=1$\\
                Solution:
                \begin{align*}
                  x' = 2&, y' = 4y \\
                  x = 2t + 3&, y = 5e^{4t} \\
                  r(1) &= \langle 5, 5e^{4} \rangle
                \end{align*}
          \item Sketch the vector field and draw the flow line starting at $(3,5)$ on the vector field \\
                Solution:
                \sageplot[height=7cm]{plot_vector_field((2, 4*y), (x, 0, 10), (y, 0, 10)) + streamline_plot((2, 4*y), (x, 0, 10), (y, 0, 10), start_points=[[3,5]])}
        \end{enumerate}
\end{enumerate}
\section{February 2 2023}
\subsection{18 -- Line Integral}
A line integral can be motivated by considering work from physics. In physics, $W = \vec{F} \cdot partialta\vec{ x}$.
\begin{defn}
  Given a vector field $\vec{F} : \mathbb{R}^{n} \to \mathbb{R}^{n}$ and a curve $C$ parametrized by the parametric equation
  $\vec{r}(t) : \mathbb{R} \to \mathbb{R}^{n}$, then the work done by the field along the curve is given by the \textbf{line integral}
  \[\int_{C} \vec{F} \cdot d\vec{r} = \int_{a}^{b} \vec{F}(\vec{r}(t)) \cdot \vec{r} '(t) dt\]
\end{defn}
\begin{ex}
  Given the vector field $\vec{F} = \langle 3, 4 \rangle$ and the path $C$ which is a line from
  $(0, 0)$ to $(0, 5)$, find the work done. \\
  Solution:
  Let $\vec{r}(t) = \langle 0, t \rangle, 0 \le t \le 5$.
  \[\int_{C} \vec{F} \cdot d\vec{r} = \int_{0}^{5} \langle 3, 4\rangle \cdot \langle 0, 1 \rangle dt
   = 20\]
\end{ex}
\begin{ex}
  Find the work done by the vector field $\vec{F}(x,y) = \langle x+ y, x \rangle$
  along the curve\\
  \sageplot[height=5cm]{plot(2*x+2, (x, -1, 0)) + plot(sqrt(4-x^2), (x, 0, 2)) + plot_vector_field((x+y, x), (x, -1, 2), (y, 0, 2))}\\
  solution: \\
  let $r_{1}(t) = \langle t,  2t + 2 \rangle, -1 \le t \le 0$
  and $r_{2}(t) = \langle 2\sin t, 2\cos t \rangle, 0 \le t \le \pi/2 $
  Then
\begin{align*}
  \int_{-1}^{0} \langle 3t + 2, t \rangle \cdot \langle 1, 2 \rangle dt
  + \int_{0}^{\pi/2} \langle 2 \sin t + 2 \cos t, 2 \sin t \rangle \cdot \langle 2 \cos t, -2 \sin t \rangle dt\\
  \int_{-1}^{0} 5t + 2 dt + \int_{0}^{2} 4 \sin t \cos t + 4 \cos^{2} t - 4 \sin^{2} t dt \\
  -(\frac{5}{2} + 2) + (1 + 1 ) = \frac{3}{2}
\end{align*}

\end{ex}
\begin{ex}
  Given the vector field $\vec{F}(x,y) = \langle 2 , x \rangle$ and the curve $C$ which is a parabola given below\\
  \sageplot[height=5cm]{plot(x^2, (x, -1, 2)) + plot_vector_field((2, x), (x, -1, 2), (y, 0, 4))}\\
  Solution: \\
  Let $\vec{r}(t) = \langle t, t^{2} \rangle, -1 \le t \le 2$
  then
  \begin{align*}
    \int_{-1}^{2} \langle 2, t \rangle \cdot \langle 1, 2t \rangle dt \\
    &= 12
  \end{align*}
\end{ex}

\section{Februrary 6 2023}
Today we are going to talk about line integrals.
\subsection{Gradient Field}
How do we determine if a given vector field is a gradient vector field?

First assume that
\begin{defn}
  We say that $\vec{F}$ is a gradient vector field iff
\[\exists \vec{F}, \vec{F} = \nabla f\]
\end{defn}
\begin{ex}
  Let $\vec{F} = \langle 2xy+1, x^2+3y \rangle$. Is $\vec{F}$ a gradient field?
  Set up the partial equations.
  \[\frac{\partial f}{\partial x} = 2xy+1 \qquad \frac{\partial f}{\partial y} = x^2+3y\]
  Integrate both sides to get
  \[f(x,y) = x^2y + x + c(y), \qquad f(x,y) = x^2y + \frac{3}{2}y^2 + c(x)\]
  Derivate the first wrt to $y$ to get
  \[f_y = x^2 + c_y(y) \implies c = 37y + c_2\]
  Then substitute that back in to get
  \[f(x,y) = x^2y + x + 37y + c_2\]
  so we may conclude that $f$ is a gradient field.
\end{ex}

\begin{ex}
  is $\vec{F} = \langle x,y\rangle$ a gradient field?
  Yes,,
  \[\vec{F} = \nabla (\frac{1}{2}x^2 + \frac{1}{2}y^2 + C)\]
\end{ex}
\begin{ex}
  Is $\vec{F} = \langle -y, x \rangle$ a gradient field??

  Assume there is such and $f$ so that
  \[f_x = -y \qquad f_y = x\]
  Integrate to get
  \[f = xy + c(x) \qquad f = -xy + c(y)\]
  Then partially differentiate to get
  \[f_x = y  + c'(x) = -y\]
  but this is a contradiction because
  \[y \neq -y\]
  so we conclude that $\vec{F}$ is not a vector field.
\end{ex}

\begin{thm}
  \textbf{Fundamental Theorem of Calculus for Line Integrals}: Given a vector field
  $\vec{F} : \mathbb{R}^n \to \mathbb{R}^n$ which is a gradient vector field (which means
  that $\vec{F} = \nabla f$) and
  given that $C$ is an oriented curve in $\mathbb{R}^n$
  from $p$ to $q$, then
  \[\int_C \vec{F} \cdot d\vec{r} = f(q) - f(p)\]
\end{thm}

You should think of this theorem as the analog of the fundamental theorem of
calculus from single variable.

\begin{ex}
  Let $\vec{F} = \langle y^2 , 2xy+1 \rangle  $.

  Find the potential function
  \[f_x = y^2 \qquad f_y = 2xy+1\]
  so \[f = xy^2 + c(y) \implies f_y = 2xt + c'(y)\]
  This means \[c'(y) = 1 \implies c(y)  =y\]
  So
  \[\vec{F} = \nabla (xy^2 + y)\]

  Now find the line integral over the curve
  \sageplot[height=5cm]{parametric_plot((x, 0), (x, -1, 0), thickness=5) + parametric_plot((0, x), (x, 0, 1), thickness=5) + plot_vector_field((y^2, 2*x*y+1), (x, -1, 0), (y, 0, 1))}
We use the FTC from before to find that it is
\[1 - (-1 \cdot 0^2 -0)  = 1\]
\end{ex}
Note that you have to pay attention to the orientation of the curve.
\begin{defn}
  A vector field $\vec{F}$ is called \textbf{path independent} (physicst
  would call it \textbf{conservative}) iff
  for all curve $C_1$ and $C_2$
  \[\int_{C_1} \vec{F} \cdot d \vec{r} = \int_{c_2} \vec{F} \cdot d \vec{r}\]
  when $c_1$ and $c_2$ have the same end-points.
\end{defn}
In other words, for a path independent vector field, the work done by the
vector field does not depend on the path taken.

\begin{defn}
  Given a vector field $\vec{F}$ and a closed curve $C$, the
  \textbf{circulation} of $\vec{F}$ along $C$ is
  \[\oint \vec{F} \cdot d \vec{r}\]
  which is just a line integral where we end up back where we started.
\end{defn}

Note that for a conservative vector field, the circulation is always $0$.

\begin{defn}
  we call a vector field $\vec{F}$ \textbf{circulation free} iff
  \[\oint_c \vec{F} \cdot d\vec{r} = 0\]
  for all closed curves $c$
\end{defn}


Here is a big theorem

\begin{thm}
  The following statements are all equivalent
  \begin{enumerate}
    \item $\vec{F}$ is a gradient field
    \item $\vec{F}$ is a path independent
    \item $\vec{F}$ is circulation free
  \end{enumerate}
\end{thm}
So if you prove one of these statements, you've proved them all,
\subsection{Curl}
\begin{defn}
  Given a vector field $\vec{F}$ and a point $(x,y)$, we defined
  the \textbf{curl} of $\vec{F}$ at $(x,y)$ to be
  \[\nabla \times \vec{F} (x,y) = \begin{vmatrix}  \frac{\partial}{\partial x} & \frac{\partial}{\partial y} \\ P & Q \end{vmatrix}= \frac{\partial Q}{\partial x} - \frac{\partial P}{\partial y}\]
  The physical interpretation is that the curl measures if you
  placed a windmill at $(x,y)$, the curl measures the how much the
  vector field will turn the windmill. If the curl is positive,
  the rotation is counter-clockwise. If the curl is negative, the
  rotation is clockwise.
\end{defn}
Another name for the curl is the \textbf{circulation density}.
\begin{ex}
  Let $\vec{F} = \langle 2xy + y^2, x^2 + x \rangle$.
  Find the curl of $\vec{F}$.\\
  \sageplot[height=6cm]{plot_vector_field((2*x*y + y^2, x^2*y + x), (x, -5, 5), (y, -5, 5))}\\
  The curl is
  \[\nabla \times \vec{F} = 1 -2x-2y + 2xy\]
\end{ex}

\begin{thm}
  \[\vec{F} \text{ is a gradient field } \implies \nabla \times \vec{F} = 0\]
  \[\nabla \times \vec{F} \neq 0 \implies  \vec{F} \text{ is not a gradient field}\]
\end{thm}

Don't misuse the theorem. The inverse of the theorem doesn't hold.
If the curl is $0$, we cannot conclude that the vector field is a
gradient field.
Consider the counter-example
\[\vec{F} = \langle \frac{-y}{\sqrt{x^2+y^2}}, \frac{x}{\sqrt{x^2+ y^2}} \rangle\]
Now find the curl
YHPL claims that the curl is $0$, but I disagree. The field is
not conservative

Here is a useful theorem to compute a line integral.
\begin{thm}\label{thm:constant_vec_f}
  If $\|\vec{F}\|$ is constant along $C$ and $\vec{F}$ is
  tangent to $C$ everywhere in the same direction, then
  \[\int_C \vec{F} \cdot d \vec{r} = \|\vec{F} \|\]
\end{thm}


\begin{ex}
  To use the previous theoerm, consider the vector field
  \[\vec{F} = \langle \frac{-y}{\sqrt{x^2+y^2}}, \frac{x}{\sqrt{x^2+y^2}}\]
  $\|F\| = 1$, and $\vec{F}$ is tangent to any circle centered at $(0,0)$.
  So, for example\\
  \sageplot[height=6cm]{plot_vector_field((-y/sqrt(x^2+y^2), x/sqrt(x^2+y^2)), (x, -3, 3), (y, -3, 3)) + parametric_plot((2*sin(x), 2*cos(x)), (x, 0, 2*pi))}\\
  Then
  \[\oint_C \vec{F} \cdot d \vec{r} = 2\pi\]
\end{ex}

How to find out if $\vec{F}$ is a gradient field.
\begin{enumerate}
  \item Find $\nabla \times \vec{F}$. If the curl is not $0$, then
    $\vec{F}$ is not a vector field. If the curl is $0$, then inconclusive
  \item Solve the differential equations to try and find a potential
    function

\end{enumerate}

\begin{thm}
  \textbf{Curl Test}:
  If $\nabla \times \vec{F} = 0 $ and the domain of $\vec{F}$ has no holes
  then
  $\vec{F}$ is a gradient vector field
\end{thm}
\section{Feb 8 2023}

Note that you cannot use the Curl Test to conclude that a given vector field is not a gradient field. The implication
goes one way. If you are applying the curl test, the conclusion is always ``the field is a gradient field''


\begin{thm}
  \textbf{Green's Theorem}:
  Let $\vec{F}$ be a smooth vector field
  Let $C$ be a smooth, closed, simple, counter-clockwise curve, and let $R$ be the region enclosed by $C$. Then
  \[\oint_{C} \vec{F} \cdot d\vec{r} = \iint_{R} \nabla \times \vec{F} dA\]
\end{thm}

The proof for Green's Theorem is that the right side of the equality counts up all the little circulations inside a region, and the
left side of the equality counts up the total circulation along the boundry of the region.
\begin{ex}
  Let $\vec{F} = \langle 2y, -2x \rangle$ and let $C$ be the closed curve.\\
  \sageplot[height=5cm]{plot_vector_field((2*y, -2*x), (x, -3, 3), (y, -3, 3)) + parametric_plot((0, x), (x, -3, 3)) + parametric_plot((3*cos(x),3*sin(x)), (x, -pi/2, pi/2))}\\
  Compute
  \[\oint_{C} \vec{F} \cdot d \vec{r}\]
  Solution: \\
  We can solve this integral in 3 different ways.
  \begin{enumerate}
  \item Using Green's Theorem
\begin{align*}
  \iint_{R} \nabla \times \vec{F} dA\\
  &= -\int_{\theta = \pi/2}^{\theta=\pi/2} \int_{r=0}^{r=3} (-4) r dr \\
  &= 18\pi
\end{align*}
Remeber to multiply by $-1$ because this curve is oriented clockwise

\item Using parametrization:
Let
\[r(t) = (0, t), -3 \le t \le 3, r(t) = (3\cos(t), 3\sin(t)), -\pi/2 \le t \le \pi/2\]
\begin{align*}
  &\int_{t=0}^{t=3} F(r(t)) \cdot r'(t) dt + \int_{t=-\pi/2}^{t=\pi/2} F(r(t)) \cdot r'(t) dt \\
  &= \int_{t=0}^{t=3} \langle 2t, 0 \rangle \cdot \langle 0, 1 \rangle dt + \int_{t=-\pi/2}^{t=\pi/2} \langle -6\sin(t) , 6\cos(t) \rangle \cdot \langle -3\sin(t), 3\cos(t) \rangle  dt \\
  &= \int_{t=0}^{t=3} 0 dt + \int_{t=-\pi/2}^{\pi/2} 18 dt \\
  &= 18\pi
\end{align*}
\item Geometric intuition:
By theorem \ref{thm:constant_vec_f}, it's just the path length, which is $3\pi$. $\vec{F}$ is perpendicular to $C$ on the vertical
seciton and $\vec{F}$ is parallel to $C$ on the circular section.
          \end{enumerate}
\end{ex}

Sometimes we can even use Greens Theorem if the curve $C$ is not closed by drawing in a new line.
\begin{ex}
  Let $\vec{F} = \langle xy + 1, x \rangle$ and let $C$ be the curve \\
  \sageplot[height=7cm]{plot_vector_field((x*y+1, x), (x, 0, 2), (y, 0, 3)) + parametric_plot((x, 3-3/2*x), (x, 0, 2))}\\
  Solution: \\
  First, by direct computation:
  Let
  \[r(t) = \langle t, 3- 3/2 t\rangle, 0 \le t \le 2\]
  \begin{align*}
    &\int_{t=0}^{t = 2} \langle 3t - 3/2 t^{2} - 1, t\rangle \cdot \langle 1, -3/2 \rangle dt \\
    &= -(3-4+2) = -1
  \end{align*}
  Now by Greens Theorem
  Draw in the other bases of the triangle.
  Let
  \begin{align*}
    &r_{1}(t) = \langle t, 0, \rangle , 0 \le t \le 3 \\
    &r_{2}(t) = \langle 0, t \rangle, 0 \le t \le 2
  \end{align*}
  Then, Green's Theorem gives
  \begin{align*}
    \oint_{C - C_{2} + C_{1}} \vec{F} \cdot d\vec{r} \\
    &= \int_{x=0}^{x=2} \int_{y=0}^{y=3-3/2x} (1-x) dy dx \\
    &= -1
  \end{align*}
  Now we compute the line integrals of $C_{1}$ and $C_{2}$ to subtract them out of the closed curve. We get
  \begin{align*}
    &\int_{C_{1}} \vec{F} \cdot d \vec{r} = \int_{t=0}^{t=2} \langle 1, t \rangle \cdot \langle 1, 0 \rangle dt = 0 \\
    &\int_{C_{2}} \vec{F} \cdot d \vec{r} \ int_{t=0}^{t=3} \langle 1, 0 \rangle \cdot \langle 0, 1 \rangle dt   = 0 \\
  \end{align*}
  The orientation/sign of curves matters
\end{ex}

We can find the area of any region if we have a function whose curl is $1$. For example
$\nabla \times \langle 0, x \rangle = 1$.
\begin{ex}
  Find the area of the region\\
  \sageplot[height=7cm]{region_plot(y <= sin(x), (x, 0, pi), (y, 0, 1))} \\
  Solution:
  By Green's Theorem
  \begin{align*}
    \int_{R} \nabla \times \vec{F} dA = \oint_{C_{1}+c_{2}} \langle 0, x \rangle \cdot d \vec{r}\\
    &= \int_{0}^{\pi} \langle 0, t, \rangle \cdot \langle 1, 0 \langle dt + \int_{0}^{\pi} \langle 0, t \rangle \cdot \langle 1 , \cos t \rangle dt
  \end{align*}
\end{ex}
\section{Feb 9 2023}
No Class today due to 12 hour meeting.
\section{Feb 13 2023}
\begin{ex}
  Let $\vec{F}(x,y) = e^{xy}(y\cos(x) - \sin(x))\vec{i} + xe^{xy}\cos(x) \vec{j}$. $C_{2}$ is the half unit circle centered at $(1,0)$
  in the first quadrant, traced clockwise from $(0,0)$ to $(2,0)$. $C_{2}$ is the line from $(0,0)$ to $(2,0)$
  \begin{enumerate}
    \item use the curl test to determine if $\vec{F}$ is a gradient field. \\
          Solution \\
          The curl test has 2 conditions. First
          \[\nabla \times \vec{F} = \frac{\partial Q}{\partial x} - \frac{\partial P}{\partial y} = 0\].
          Next the domain condition.
          The domain has no holes.
          By the curl test, $\vec{F}$ is a gradient field
    \item Find the potential function of $\vec{f}$
          Solution: \\
          Set up some differential equations.
          \[f_{x} = e^{xy})(y\cos(x) - \sin(x)) \qquad f_{y} = xe^{xy}\cos(x)\]
          So
          \[f = \cos(x) e^{xy} + K(x) \]
          \[f_{x} = -y\sin(x)e^{xy} + K'(x)\]
          so
          \[f = e^{xy}\cos(x)\]
    \item Set up the line integral $\int_{C_{1}} \vec{F} \cdot d \vec{r}$ using
          parametrization, do not evaluate. \\
          Solution: \\
          Let
          \[\vec{r}(t) = \langle 1 -\cos(t), \sin(t)  \rangle, 0 \le t \le \pi\]
          \[\int_{t=0}^{t=\pi} \vec{F}(\vec{r}(t)) \vec{r}'(t) dt\]
    \item Use the fundamental theorem of line integrals to evaluate $\int_{C_{1}} \vec{F} \cdot d\vec{r}$\\
          This theorem says we just eval at the beginning and the end
          \[\int_{C_{1}} \vec{F} \cdot d \vec{r} = f(2,0) - f(0,0) = \cos(2) - 1\]
    \item evaluate $\int_{C_{2}} \vec{F} \cdot d\vec{r}$ using a parametrisation.\\
          Let
          \[r(t) = \langle t, 0 \rangle 0 \le t \le 2\].
          \[r'(t) = \langle 1, 0 \rangle 0 \le t \le 2\]
          \[\int_{t=0}^{t=2} \langle -\sin(t),t\cos(t) \rangle \cdot \langle 1, 0\rangle dt\]
    \item is $\vec{F}$ conservative? \\
          Yes $\vec{F}$ is conservative because $\vec{F}$ is a gradient vector field.
    \end{enumerate}
  \end{ex}
  We already did the chapter 17 KC sample.
  \begin{ex}
    Let $C$ be the curve from $(-1,0)$ to $(1, 0)$ on the curve $y=  1-x^{2}$.
    Let $\vec{F} = \langle 2, e^{y^{2022}} + x^{2} + 3$. Use Greens Theorem to
    find the work done by $\vec{F}$ on $C$. \\
    Solution: \\
    We need to close the curve in order to apply Greens Theorem.
    Note that $\nabla \times \vec{F} = 2x$
    \[\oint_{C_{2}-C_{1}} \vec{F} \cdot d\vec{r} = \iint 2x dA  \]
    The area integral is
    \[\int_{x=-1}^{x=1} \int_{y=0}^{y=1-x^{2}} 2x dy dx = \int_{-1}^{1} 2x(1-x^{2}) dx = 0\]
    \[\int_{C_{1}} \vec{F} \cdot d\vec{r} =\int_{C_{1}} \vec{F} \cdot d\vec{r} \]
    Let $\vec{r}(t) = \langle t, 0 \rangle -1 \le t \le 1$
    \[\int_{t=-1}^{t=1} \langle 2, t^{2} + 4 \rangle \cdot \langle 1, 0 \rangle dt = 2 \]
  \end{ex}

  \begin{ex}
    Let $\vec{F} = \langle 2 , e^{y^{2023}} + x^{2} + 3 \rangle $
    Let $C$ be the clockwise region bounded by $y=1$, $x=0$, $y=x$.
    Find the clockwise circulation of the vector field around the boundry.
    \[\int_{0}\]
  \end{ex}



  Here we did some problems out of the textbook, and I couldn't type
  them up because they didn't give the equation.
  \subsection{Chapter 19 - flu}
  Chapter 19 is about flux.
  flux is the rate of flow of a vector field through a surface $S$. It is
  written $\int_{S} \vec{F} \cdot d\vec{A}$, where $\vec{A}$ is vector normal
  to $A$ whose length is equal to the area of $A$.
\section{Chapter 18 Knowledge Check Solutions}
\begin{enumerate}
  \item Consider $F(x,y) = \langle y + 1, x \rangle$ and $G(x,y) = \langle \frac{-y}{\sqrt{x^{2}+y^{2}}}        , \frac{x}{\sqrt{x^{2}+y^{2}}}$
        \begin{enumerate}
          \item Does the curl test apply to $F$?\\
                Solution: \\
                Yes, the curl test applies. The $\nabla \times F = 0$ and
                $F$ has no holes in its domain.
          \item Is $F$ conservative? \\
                Solution: \\
                Yes, $F$ is conservative because of the Curl Test.
          \item Apply Fundamental Theorem of Calculus to evaluate $\int_{C} \vec{F} \cdot d\vec{r}$
                where $C$ is the curve from $(01, e^{-1})$ to $(1, e)$ along
                $y = e^{x}$\\
                Solution:\\
                We find that if $\nabla f = \vec{F}$, then $f(x,y) = xy + x + c$
                . The FTC says the answer is $f(1,e) - f(-1, e^{-1})$, which is
                $e+e^{-1}+2$
          \item Does the curl test apply to $\vec{G}$?\\
                Solution: \\
                No, $\vec{G}$ has a hole at $(0,0)$
          \item Find the work done by $\vec{G}$ along the circle of radius $2$
                at the origin going counterclockwise, starting at the point $(2,0)$\\
                \begin{enumerate}
                  \item using explicit parametrization of $C$ \\
                        Solution: \\
                        let
                \[r(t) = \langle 2\cos t, 2\sin t \rangle \qquad  0 \le t \le 2\pi\]

                        \begin{align*}
                          \oint_{C} \vec{G} \cdot d\vec{r} &= \int_{0}^{2\pi} \langle -\sin t, \cos t \rangle \cdot \langle -2 \sin t , 2\cos t \rangle dt\\
                                                           &= \int_{0}^{2\pi} 2 dt \\
                          &= 4\pi
                          \end{align*}
                  \item using geometric intuition about $\vec{G}$ and $C$ \\
                        Solution: \\
                        $\vec{G}$ is always perpendicular to $C$, so it's just
                        \[\|\vec{G}\| |c| = 1 4\pi = 4\pi\]
                \end{enumerate}
          \item is $\vec{G}$ conservative?
                \\Solution:\\
                No, because in the integral above, the circulation was not $0$.

        \end{enumerate}
        \item Let $\vec{F}(x,y) = \langle 2, e^{y^{2023}} +x^{2}+3 \rangle$
        \begin{enumerate}
          \item Find the clockwise circulation of $\vec{F}$ around the region bounded
                by $x=0$, $y=1, y=x$ \\
                Solution: \\
                Just use Greens Theorem. Make sure to multiply by $-1$ because we
                are going clockwise.
                \begin{align*}
                  &\oint_{R} \vec{F} \cdot d\vec{r} \\
                  &= \iint \nabla \times \vec{F} dA \\
                  &= \int_{x=0}^{x=1}\int_{y=x}^{y=1} 2x dy dx \\
                  &= -1/3
                  \end{align*}
                \item Use Green’s theorem to find the work done by the vector field along a curve $C$, where $C$ is from $(-1, 0)$ to $(1, 0)$ along $y=1-x^{2}$.
                \\
                Soltion: \\
                You need to close the loop by drawing in an extra line. I suggest defining $C_{2}$ to be the line from $(-1, 0)$
                to $(1, 0)$. Then, let $R$ be the region enclosed by $C_{2}-C$By Green's Theorem
                \begin{align*}
                  \oint_{R} \vec{F} \cdot d\vec{r} &= \iint \nabla \times \vec{F} dA \\
                  \int_{C_{2}} \vec{F} \cdot d\vec{r} - \int_{C} \vec{F} \cdot d\vec{r} &= \int_{-1}^{1} \int_{0}^{1-x^{2}} 2x dy dx \\
                  \int_{C_{2}} \vec{F} \cdot d\vec{r} &= \int_{C} \vec{F} \cdot d\vec{r} \\
                  \int_{C} \vec{F} \cdot d\vec{r} = \int_{t=-1}^{t=1} \langle 2, t^{2} + 3 \rangle \cdot \langle 1 , 0 \rangle dt\\
                  &=4
                \end{align*}
                \end{enumerate}
\end{enumerate}
\section{Feb 16 2023}
\subsection{19.2 -- Flux}
\begin{defn}
  The \textbf{flux} of a vector field through a surface is the amount of flow of a vector field
  through the surface.
  $S$ is the surface, $\vec{F}$ is the vector field, and $\vec{n}$ is the normal vector to the surface.
  \[\Phi = \iint_{S} \vec{F} \cdot d\vec{S} = \iint_{S} \vec{F} \cdot \vec{n} dS\]
  In this notation, $d\vec{A} = \vec{n}dA$
\end{defn}

There are several easy cases if we know the surface.
You need to memorize these $3$ cases:
\begin{enumerate}
  \item cylinder
  \item sphere
        \item $z = f(x,y)$
\end{enumerate}
\subsubsection{Cylinder}
\begin{ex}
  Let $S$ be a cylinder whose axis is the $z$ -axis. There is no top/bottom cap, $S$ is only the
  sides of the cylinder.
  \begin{sageblock}
    from sage.plot.plot3d.shapes import Cylinder
    \end{sageblock}
  \sageplot[height=7cm]{Cylinder(1, 1, closed=False) + arrow((0, 1, 0.5), (0, 2, 0.5))}
  The unit normal vector $\vec{n}$ for the point with cylindrical coordinates $(r, \theta, z)$ is
  $\langle \cos \theta, \sin \theta , 0 \rangle$.

  We can use the unit normal to calculate the flux.
  \[\Phi = \int_{\theta = 0}^{\theta = 2\pi} \int_{z=a}^{z=b}   \,dz d\theta\]
\end{ex}

Here is an example where the vector field is given
\begin{ex}
  Let $S$ be the surface which is the sides of a cylinder whose axis is the $y$-axis from
  $y=-2$ to $y=3$ and whose radius is $4$.

  Let $\vec{F}(x,y) = \langle y, x+1, z \rangle $.
  find the flux of $\vec{F}$ through $S$.\\
    \sageplot[scale=0.5]{implicit_plot3d(x^2 + z^2 == 4^2, (x, -4, 4), (y, -2, 3), (z, -4, 4)) + plot_vector_field3d((y, x+1, z), (x, -4, 4), (y, -2, 3), (z, -4, 4))}\\
  We can find the flux by
  \begin{align*}
    \Phi \\
    &= \int_{y=-2}^{y=3} \int_{\theta = 0}^{\theta = 2\pi} \langle y 4\cos \theta, 4\sin \theta \rangle \cdot \langle -\cos\theta, 0, -\sin\theta \rangle d\theta dy
  \end{align*}
\end{ex}
\subsubsection{Sphere}
For a sphere, use spherical coordinates.

\[\Phi = \int_{\theta = 0}^{\theta = 2\pi} \int_{\phi = 0}^{\phi = 2\pi} \vec{F} \cdot \langle \sin\phi\cos\theta, \sin \phi \cos \theta, \cos \phi \rangle d\phi d\theta\]

Note that a top hemisphere only means the top half of the sphere
\begin{ex}
  Let $S$ be the top hemisphere and let $\vec{F} = \langle 1,z, y \rangle$. Find the flux.\\
    \sageplot[scale=0.7, trim={5cm 4cm 5cm 2cm}]{implicit_plot3d(x^2 + z^2  + y^2 == 2^2, (x, -2, 2), (y, -2, 2), (z, 0, 2)) + plot_vector_field3d((1, z, y), (x, -2, 2), (y, -2, 2), (z, 0, 2))}
  \begin{align*}
    \Phi \\
    &= \int_{\theta = 0}^{\theta = 2\pi} \int_{\phi = 0}^{\phi = \pi / 2} \langle 1, 2\cos\phi, 2\sin\phi\cos\theta \rangle \cdot
 \langle \sin\phi\cos\theta, \sin \phi \cos \theta, \cos \phi \rangle d\phi d\theta\
  \end{align*}
\end{ex}
Note that we often set up integrals, but never evaluate them.
\subsubsection{Implicit Surfaces}
If we are given a surface as $S$ defined by $z = f(x,y)$, then
\[\vec{n} = \langle -f_{x}, -f_{y}, 1 \rangle \]

Where $1$ means that $S$ is oriented upwards and $-1$ means $S$ is oriented downwards.

In general, we project the surface onto a plane.

\begin{ex}
  Let $S$ be the surface $y = \sqrt{x^{2}+y^{2}}$ for $0 \le y \le 2$ where the normal vector is oriented outwards. Let $\vec{F} = \langle 1+z, x, y \rangle$.
  Find the flux of $\vec{F}$ through $S$. Note that $S$ is a cone.
  \\
  \sageplot[scale=0.5]{implicit_plot3d(y==sqrt(x^2+z^2), (x, -3, 3), (y, 0, 2), (z, -3, 3) ) + plot_vector_field3d((1+z, x, y), (x, -3, 3), (y, 0, 3), (z, -3, 3))}\\
  We have that $\vec{n} = \langle f_{x}, -1, f_{z} \rangle$. Note that we use $-1$ because when we project onto $x-z$ plane, the normal vector is pointing down.
  \begin{align*}
    \Phi
    &= \int_{-2}^{2} \int_{-\sqrt{4-x^{2}}}^{\sqrt{4-x^{2}}}\langle 1+z, x , \sqrt{x^{2}+z^{2}} \rangle \cdot \langle f_{x}, -1, f_{z} \rangle dxdz\\
    &= \int_{} \langle 1+z, x , \sqrt{x^{2}+z^{2}} \rangle \cdot \langle \frac{1}{2\sqrt{x^{2}+z^{2}}}, -1, \frac{1}{2\sqrt{x^{2}+z^{2}}} \rangle dxdz\\
      &= \int_{\theta= 0}^{\theta = 2\pi}\int_{r=0}^{r=2} \langle 1+r\sin\theta, r\cos\theta, r \rangle \cdot \langle \frac{r\cos\theta}{r},1 \frac{r\sin\theta}{r} \rangle rdrd\theta
   \end{align*}
  \end{ex}


  Here is a useful theorem to compute flux integral.
  \begin{thm}
    \textbf{Geometric Intuition Theorem}: If $\vec{F}$ is always perpendicular to $S$ and $\|\vec{F}\|$ is constant
    on $S$, then
    \[\Phi_{S} = \|\vec{F}\|A\]

    \end{thm}

    \begin{ex}
      Let $\vec{F}(x,y,z) = \langle x,y,z \rangle$ and let $S$ be the surface which is the upper hemisphere of a sphere with radius 2.\\

    \sageplot[scale=0.7, trim={5cm 4cm 5cm 2cm}]{implicit_plot3d(x^2 + z^2  + y^2 == 2^2, (x, -2, 2), (y, -2, 2), (z, 0, 2)) + plot_vector_field3d((x, y, z), (x, -2, 2), (y, -2, 2), (z, 0, 2))}\\
      By the geometric intuition, the answer is $\|\vec{F}\|A = 2\pi2^{2}2 = 16\pi$
      \end{ex}

      You can parametrize a surface by a function
      \[r(s,t) = \langle x(s,t), y(s,t), z(s,t) \rangle \]
      \section{Feb 20 2023}
      No class due to holiday.
      \section{Feb 22 2023}
      \subsection{Review}
      First, let's review what we learned about the flux from last time.
      \begin{enumerate}
        \item the flux is defined as $\Phi = \oint_{S} \vec{F} \cdot d\vec{A}$
        \item If $S$ is a surface defined by $z = f(x,y)$, then $d\vec{A} =
              \langle -f_{x}, -f_{y}, 1 \rangle $
        \item if the magnitude of $\vec{F}$ is constant, and $\vec{F}$ is
              always perpendicular to $S$, then $\Phi = \|\vec{F}\||S| $
              \item depending on the surface, it may be more convient to use cylindrical or spherical coordinates
              \item pay attention to the orientation of the surface (you may need to multiply by $-1$)
      \end{enumerate}
      \begin{ex}
        Set up flux integral for $\vec{F} = \langle y, 2x+1, y+z \rangle $ and $S$ is the following surfaces. Here are $6$ examples
        \begin{enumerate}
               \item $S$ is bounded by
        $y = 2-(x^{2} + z^{2})$ and $y \ge 0$. $S$ is oriented inwards.\\
        \sageplot[scale=0.5]{implicit_plot3d(y == 2-x^2-z^2, (x, -3, 3), (y, 0, 2), (z, -3, 3)) + plot_vector_field3d((y, 2*x+1, y+z), (x,-3, 3), (y, 0, 2), (z, -3, 3))}\\
                Solution: \\
                \[d\vec{A} = \langle 2x,-1,2z\rangle \]
                \begin{align*}
                  \Phi = \int_{x=-\sqrt{2}}^{x=\sqrt{2}} \int_{z=\sqrt{2-x^{2}}}^{z=\sqrt{2-x^{2}}} \langle 2-(x^{2}+z^{2}), 2x+1, 2-(x^{2}+z^{2})+z\rangle\cdot\langle-2x, -1, -2z \rangle dzdx
                  \end{align*}
          \item $S: x^{2}+y^{2} = 9$ $-1 \le y \le 2$, oriented outwards \\

                Solution: \\
                This is just a cylinder
                \begin{align*}
                  \Phi  = \int_{\theta= 0}^{\theta=2\pi}\int_{y=-1}^{y=2} \langle y,2(3\cos\theta+1), y + 3\sin\theta \rangle \cdot \langle \cos\theta, 0, \sin\theta \rangle 3 dy d\theta
                  \end{align*}
          \item $S: x = \sqrt{9-x^{2}-z^{2}}$, oriented inward\\
                Solution: \\
                This is a half-hemisphere of radius $3$.
                \begin{align*}
                  \Phi  = \int_{0}^{2\pi}\int_{0}^{\pi} \langle 3\cos\theta\sin\phi, 2(3\sin\phi\cos\theta) + 1, 3\sin\phi\sin\theta + 3\cos\theta \rangle\\ \cdot \langle -\sin\phi\cos\theta, -\sin\phi\cos\theta, -\cos\phi \rangle 9\sin\phi d\phi\theta
                  \end{align*}
          \item $S: y=1$ over $[-1, 2] \times [3, 4]$ oriented towards positive $y$.

                \begin{align*}
                  \Phi  = \int_{x=-1}^{x=2} \int_{z=3}^{z=4} \langle 1,2x+1, 1+z \rangle \cdot \langle 0 , 1, 0 \rangle dz dx
                  \end{align*}
                \item $S : $ equilateral triangle with vertices at $(1,0,0), (0,2,0), (0,0,3)$ oriented towards the origin.
The equation of the surface is $\frac{x}{1} + \frac{y}{2} + \frac{z}{3} = 1$. The normal vector is constantly $\vec{n} = \langle -6, -3, -2 \rangle$
                \begin{align*}
                  \Phi  = \int_{x=0}^{x=1} \int_{y=0}^{y=2-2x}\langle y 2x+1, y + 3 - 2x - 3/2 y \rangle \cdot \langle -3 -3/2, -1 \langle dy dx
                  \end{align*}
                \item $S: y^{2} + z^{2} \le 16$ on $x=5$ oriented downward.
 $S$ is a disk.
                \begin{align*}
                  \Phi  = \int_{\theta = 0}^{\theta = 2\pi} \int_{r=0}^{r=4}\langle r\cos\theta, 2(5)+1, r\cos\theta + r\sin\theta \rangle \cdot \langle -1, 0, 0 \rangle r dr d\theta
                  \end{align*}
                \end{enumerate}
    \end{ex}
    \subsection{19.3 -- Divergence}

    \begin{defn}
      The \textbf{divergence} is the flux density. If the divergence is positive, new fluid springs into existance at that point. If the
      divergence is negative, then fluid disappears at that point. It can be defined as the flux through a very small surface divided by the volume of the region enclosed which encloses
      that point. We might write
      \[div \vec{F} = \lim_{V(W) \to 0} \frac{\oint_{S} \vec{F} \cdot d\vec{A}}{V(W)}\]
      \end{defn}
      \begin{ex}
        Let $\vec{F} = z, 1+2y, x+3 \rangle$ find divergence of $\vec{F}$ at the origin. We can use any surface to find the diverence.
        $S_{1} : z= \sqrt{x^{2}+y^{2}}$ or $S_{2} :$ cylinder from $ a  < y < b$ or $S_{3} : $

        We will do 2 examples
      \end{ex}
      \begin{thm}
        The divergence does not depend on what surface you choose. It only depends on the point $p$ and field $\vec{F}$
      \end{thm}
      \section{Feb 23 2023}
      \begin{thm}
        \textbf{Divergence Theorem}
        If $S$ is a closed surface with interior region $E$, then $\Phi$, the flux through $S$ can be calculated as
        \[\Phi = \oint_{S} \vec{F} \cdot d\vec{A} = \iiint_{E} \nabla \cdot \vec{F} dV \]
        \end{thm}
      Let's start off with some examples to review divergence.
      \begin{ex}
        $\vec{F} = \langle e^{\cos (yz^{2023})} + x^{2}, 3y + \sin(e^{x^{2023}}) , xyz \rangle $.  Let $S_{1} : z =x^{2}+y^{2}$ and
        $S_{2}: z = 4$. Let $S = S_{1} + S_{2}$ where $S$ is oriented inwards. Find the flux. \\
        Solution: \\
        By the divergence theorem
        It's easier in polar coordinates
        \begin{align*}
          \Phi
          &= \iiint_{E} \nabla \cdot \vec{F} dV \\
          &= -\int_{0}^{2\pi}\int_{r=0}^{r=2}\int_{z=r^{2}}^{z=4}(2r\cos\theta + 3)r dz dr d\theta\\
          &= -\int_{0}^{2\pi}\int_{0}^{2}(4-r^{2})(2r\cos\theta + 3)r dz dr d\theta\\
            &= -24\pi
        \end{align*}
      \end{ex}
      Now a simple example
      \begin{ex}
        Let $S$ be the equilateral triangle in the 1st octant with veritices at $(2,0,0), (0,2,0), (0,0,2)$
        oriented outwards. Let $\vec{F} = 3\langle x,y,z\rangle$. Note that the surface is NOT closed, so in order
        to use the divergence theorem, we must close the surface. \\
        Solution: \\
        \sageplot[scale=0.5]{list_plot3d([(2,0,0), (0,2,0), (0,0,2)]) +  plot_vector_field3d((3*x, 3*y, 3*z), (x,-2, 2), (y, 0, 2), (z, 0, 2))}\\
        We can use either direct computation of the flux, or use the divergence theorem if we close the solid
        ourselves.

        First, direct computation.

          $\frac{x}{2} + \frac{y}{2} + \frac{z}{2} = 1 \implies z = 2-x-y$
        \begin{align*}
          \Phi
          &= \int_{x=0}^{x=2}\int_{y=0}^{y=2-x} 3\langle x,y, 2-x-y \rangle \cdot \langle 1, 1, 1 \rangle dy dx \\
          &= \int_{x=0}^{x=2}\int_{y=0}^{y=1-x} 3(x + y + 2 - x - y) dy dx \\
          &= 12
        \end{align*}
        Now, the divergence theorem. Close the solid as a triangular pyramid.
        \begin{align*}
          \Phi_{S}
          &= \int_{x=0}^{x=2} \int_{y=0}^{y=1-x}\int_{z=0}^{z=1-x-y} 9 dz dy dx \\
          &= 9V\\
          &= 12
        \end{align*}
        Now subtract away the other 3 surfaces (we aren't done yet!)
        \begin{align*}
          \Phi_{S_{1}+S_{2}+s_{3}} = 0
          \end{align*}
        You could have evaluated this using geometric intuition because $\vec{F} \cdot d\vec{A}$ is constant, but I won't do that.
      \end{ex}
      \begin{ex}
        A greenhouse is in the shape of the graph $z=9-x^{2}-y^{2}$ with the floor at $z=0$. Suppose the temperatutre around
        the greenhouse is given by $T(x,y,z) = 2x^{2}+2y^{2} + (z-3)^{2}$. Let $\vec{H} = -\nabla T$ be the heat flux density
        field.

        Use the divergence theorem to calculate the total heat flux outward across the boundary wall of the greenhouse? \\

        Solution: \\
        $-\nabla T = -\langle 4x, 4y, 2z-6 \rangle $. $\nabla \cdot \vec{H}  = -10$
        \begin{align*}
          \Phi
          &= \int_{\theta= 0}^{\theta = 2\pi}\int_{r=0}^{r=3}\int_{z=0}^{z=9-r^{2}} -10 r dz dr d\theta \\
          &= -405\pi
        \end{align*}
        We now have to subtract the floor (we only want the wall).
        \begin{align*}
          \Phi_{floor}
          &= \int_{0}^{2\pi}\int_{0}^{3} \langle -4r\cos\theta, -4r\sin\theta, -2(0-3) \rangle \cdot \langle 0, 0, -1 \rangle r dr d\theta\\
          &= -54\pi
        \end{align*}
      \end{ex}
      \subsection{Chapter 20}
\end{document}
