\documentclass[11pt]{article}
%  \usepackage{evan}
\usepackage{mathtools}
\usepackage{sagetex}
\usepackage{amsfonts}
\usepackage{amsmath}
%\usepackage{sectsty}
% \usepackage{graphicx}
\newtheorem{thm}{Theorem}
\newtheorem{defn}{Definition}
\newtheorem{ex}{Example}
% Margins
\topmargin=-0.45in
\evensidemargin=0in
\oddsidemargin=0in
\textwidth=6.5in
\textheight=9.0in
\headsep=0.25in

\title{Multi Notes}
\author{Atticus Kuhn}
\date{\today}


\begin{document}
\maketitle
\tableofcontents

% Optional TOC
% \tableofcontents
% \pagebreak

%--Paper--

\section{Jan 11. 2023}
\subsection{Syllabus}
Learning opportunity is a waste of time.
Teach your parents stuff, YHPL is not happy (it is meaningless).
Each knowledge check is worth more (20\%).
\subsection{Overview of 1D}
We are about to complete calculus.
You want to take real analysis or complex analysis after 1D.
\begin{table}[h]
  \centering
  \begin{tabular}{c|c}
    Class & Topic \\
    1A & single variable derivatives \\
    1b & single variable integrals \\
    1C & multi-variable derivatives \\
    1D & multi-variable integra;s \\
    \end{tabular}
  \end{table}



\subsubsection{Topics Covered in 1D}
Topics covered in 1D:
\begin{enumerate}
  \item work
\item line integral = work done by force
  \item vector fields
\item flux integral = rate of flow thru a thin net
  \item Chapter 20: Green's Theorem, curl, divergenence
    \item That's the end
\end{enumerate}
If we meet YHPL, she will bake us a cake.
\subsection{Review from 1C}
\subsubsection{review of quartic surfaces}
Review quadric surfaces
\begin{table}[h]
  \centering
  \begin{tabular}{||c c||}
    eqn & name \\
    $z = x^2 - y^ 2$ & Hyperbolic paraboloid \\
    $ z = y^2$ & Parabolic cylinder \\
    $z = \sqrt{4-x^2-y^2}$ & half-hemisphere of a sphere\\
    $z = \sqrt{x^2+y^2}$ & elleptic cone \\
    $ z = x^2 + y^2 $ & elliptic paraboloid \\
    $ z  =1 - \sqrt{x^2 + y^2 } $ & elliptic cone \\
    $ z = 6-3x-2y$ & plane \\
    $ z = 6-2y$ & plane thru $(37.0.6)$ and normal vector is $\vec{n} = \langle 0 ,2 ,1 \rangle$ \\
    $z = 4-x^2-y^2$ &  Elliptic Paraboloid\\
   \end{tabular}
  \end{table}
Review table that YHPL posted on quartic surfaces.
\subsubsection{review of integrals}
Recall that the definition of integral is
\[\int_I f(x) \, dx = \int_a^b f(x) \, dx = \lim_{n \to \infty} \sum_{i=1}^n f(x_i^*) \Delta x_i\]
where $f(x_i^*)$ is the height of the $i^{th}$ rectangle and $\Delta x_i$ is the width of the
$i^{th}$ rectangle. We are summing up areas on many small rectangles. Note that area can be
negative if the function goes below the x-axis (called signed area).
Note that all the $\Delta x$s do not all have to be the same width.

\begin{thm}
  Average
$avg = \frac{1}{|I|} \int_I f(x) \, dx$
\end{thm}
\subsection{Multi-Variable integration}
Given that $z = f(x,y)$, we find the area by splitting the region into rectangles under the
curve. We split the x-axis and we split the y-axis. We integrate over a region $R$.
Adding up the volume of all the little rectangular prisms approximates the volume of
our original curve.

\begin{defn}
  The definition of a \textbf{multi-variable} integral is
  \[\iint_R f(x,y) \,dx \,dy = \int_R f(x,y) dA = \lim_{n\to \infty} \lim_{n \to \infty} \sum_{i=1}^n \sum_{j=1}^m f(x_i^*, y_j^*) \Delta x_i \Delta y_j\]
\end{defn}
In a double integral, you have to integral twice.
\subsubsection{numerical approximation in multiple variables}
Let's use this table as an example of a function

\begin{table}
  \centering
  \begin{tabular}{c c c c c}
    x y &1 &4& 7 &10\\
    4 & 5 & 6 & 9 & 37 \\
    2 & 11 & 14 & 7 & 12 \\
    0 & 16 & 21 & 0 & 15
  \end{tabular}
\end{table}
Let $R = [1,10] \times [[0,4]]$.
\[\int_R f dA \approx (14 + 6 + 9 + 37 + 12 + 7)\cdot3\cdot 2\]

Example: Let $z = e^{-(x^2+y^2)}$. If you sample the function using the bottom left approximation,
then this is an overestimate because the rest of the rectangle with have higher $z$-values.

Given a contour map, we can approximate an integral. Divide up the graph and pick a point
from each division to represent the whole.
\subsection{Summary of surfaces}
\begin{sageblock}
var('x y z')
\end{sageblock}
\begin{table}[h]
  \centering
  \begin{tabular}{c c c }
    Name & Equation & Graph \\
    Ellipsoid  & $\frac{x^2}{a^2} + \frac{y^2}{b^2} + \frac{z^2}{c^22} = 1$ &  \sageplot[height=3cm]{implicit_plot3d(x^2/2+y^2/3  + z^2/4 == 1 , (x,-2, 2), (y, -2, 2), (z, -2, 2)),frame=False} \\
    Cone & $\frac{x^2}{a^2} + \frac{y^2}{b^2} = \frac{z^2}{c^2}$ &   \sageplot[height=3cm]{implicit_plot3d(x^2/2+y^2/3 == z^2/4 , (x,-2, 2), (y, -2, 2), (z, -2, 2),frame=False)} \\
    Elliptic Paraboloid & $\frac{x^2}{a^2} + \frac{y^2}{b^2} = \frac{z}{c}$&   \sageplot[height=3cm]{implicit_plot3d(x^2/2+y^2/3 == z/4 , (x,-2, 2), (y, -2, 2), (z, -2, 2),frame=False)} \\
    Hyperboloid of One Sheet & $\frac{x^2}{a^2} + \frac{y^2}{b^2} - \frac{z^2}{c^2} = 1$ &   \sageplot[height=3cm]{implicit_plot3d(x^2/2+y^2/3 - z^2/4==1 , (x,-2, 2), (y, -2, 2), (z, -2, 2),frame=False)} \\
    Hyperboloid of Two Sheets & $\frac{x^2}{a^2} - \frac{y^2}{b^2} = \frac{z}{c}$ &   \sageplot[height=3cm]{implicit_plot3d(x^2/2+y^2/3 == z/4 , (x,-2, 2), (y, -2, 2), (z, -2, 2),frame=False)} \\
    Hyperbolic Paraboloid & $-\frac{x^2}{a^2} - \frac{y^2}{b^2} + \frac{z^2}{c^2}$=1 &   \sageplot[height=3cm]{implicit_plot3d(-x^2/2-y^2/3 + z^2/4==1 , (x,-2, 2), (y, -2, 2), (z, -2, 2),frame=False)} \\
    \end{tabular}
  \end{table}
\section{January 12, 2023}
no class due to doctor's appointment
\section{January 15, 2023}
No class due to MLK day
\section{January 18, 2023}
If we are integrating over a non-rectangular region, the inner bounds must depend on the outer bounds.
\subsection{Integrating non-rectangular regions}
Example:

A strange region may be given by a triangle an semi-cirle

\[\int_{x=-1}^{x=0} \int_{y=1}^{y=x+2} f dydx + \int_{x=0}^{x=\sqrt{3}} \int_{y=2}^{y=\sqrt{4-x^2}} f dy dx\]
But we could also do as y

\[\int_{y=1}^{y=2} \int_{x=-1}^{x=y-2} f dx dy + \int_{y=1}^{y=2} \int_{x=0}^{x=\sqrt{4-y^2}} f dx dy\]
Note that it could be (no need to split)
\[\int_{x=1}^{x=2} \int_{y=1}^{y=\sqrt{4-x^2}} f dx dy\]
Excercise:
Given the integral
\[\int_{-1}^0 \int_1^{4-x} f dy dx\]
reverse the order of the integrals
\[\int_{y=1}^{y=4} \int_{x=-1}^{x=0} f dx dy  + \int_{y=4}^{y=5} \int_{x=-1}^{x=4-y} f dx dy\]

\subsection{Application of Double Integrals}
We use double integrals to find volume.
\begin{ex}
Exercise: given this weird parabola, find volume.
\sageplot[height=5cm]{implicit_plot3d(z == 25 - x^2 - y^2, (x, -3, 3), (y, -3, 3), (z, 16, 25))}
\[s_1: z= 25 - x^2 - y^2, s_2: z = 16\]

To find volume
\[\int_{x=-3}^{x=3} \int_{y=-\sqrt{9-x^2}}^{y=\sqrt{9-x^2}} 9 - x^2-y^2 dy dx\]
Pro =-Tip: always project orthoganlly onto the xy plane
\end{ex}

Another example
\begin{ex}
Given the mass-density of
\[s(x,y) = \sqrt{x^2+y^2}\]
the mass of a triangle is given by
\[\int_{x=0}^{x=2} \int_{y=0}^{y=4-2x} s(x,y) dy dx\]
\end{ex}
\begin{ex}
Example:
Given a city with the shape of a semi-circle. Find the average distance
from the city to the ocean.

YHPL strongly recommends reading ahead.
\sageplot[height=5cm]{plot(sqrt(9-x^2), (x, -3, 3)) + plot(0, (x, -3, 3))}

The average distance from the city to the ocean is given by
\[\frac{\int_{x=-3}^{x=3} \int_{y=0}^{y=\sqrt{9-x^2}} y \, dy dx}{\int_{x=-3}^{x=3} \int_{y=0}^{y=\sqrt{9-x^2}} 1 \, dy dx}\]

To get the average distance to the ocean, we get the total distance and
then divide by the total area.
\end{ex}

\begin{ex}
Without computation, find the sign of
\[\int_R (y^3-y) dA\]
and
\[\int_R e^yx dA\]
where
the region $R$
is
\sageplot[height=5cm]{plot(-sqrt(1-x^2), (x, -1, 1))}?

Solution:
Remember the definition. The first integral will be negative because
$y < 0 \implies y^3 > y \implies y^3 - y > 0$.
The second integral will be 0, because the negative $x$ perfectly cancels out the positive $x$.

\end{ex}

\begin{ex}
 What is the sign of
 \[\int_R cos(x) dA\]
 where $R$ is the same region as above?
 Solution: It is positive because $\cos{x} > 0$ for $-1 < x < 1$. Cosine only becomes 0 at $\pi/2 \approx 1.5$.
\end{ex}


\subsection{Triple Integral}
\begin{defn}
  Given a funciton $w = f(x,y,z)$ and a region $W \subset \mathbb{R}^3$
  \[\iiint_w f(x,y,z) dV\]
\end{defn}
There are $3! = 6$ different ways to integrate a triple integral. Where the inner integrals can depend on
the outer integrals.

\begin{ex}
  Redo the parabola volume question using a triple integral.
  \[\int_{x=-3}^{x=3} \int_{y=-\sqrt{9-x^2}}^{y=\sqrt{9-x^2}} \int_{z=16}^{z=25-x^2-y^2} 1 \, dz dy dx\]
\end{ex}


\section{January 19, 2023}
\subsection{Review}
Review of what we learned last week
\begin{enumerate}
  \item find volume by integrating over 1
  \item determine sign of integral by using sign of integrand over region
  \item You can swap the order of integration (either $dx dy$ or $dy dx$), and sometimes
    one order of integration will be easier than the other
  \item When approximating a double integral, you pick a point from each region to approximate the
    entire region. From a contour map, you can choose the sample point and then multiply that
    by the area of the region.
  \item you can find average function value by taking double integral over region and
    then divide by the area of that region.
\end{enumerate}
\begin{ex}
  Here's an application problem:
  Estimate the average snowfall in Colorado based on this map. Sample based on the midpoint of each rectangular region.
  \[\frac{1}{16}(16+16+19+13+8+28+18+13+2+24+17+11+0+16+8+7) = \frac{27}{2}\]
\end{ex}
\subsection{16.4 -- Polar Coordinates}
In polar coordinates, $(r, \theta)$, a point is represented by its distance from the origin, $r$, and the angle it makes with the positive $x$-axis, $\theta$.


\begin{ex}
Let's revisit the problem of the average distance from the city to the ocean that we did
last week.
The same integral becomes
\begin{align*}
\int_{\theta = 0}^{\theta = \pi} \int_{r=0}^{r=3} r^2 \sin(\theta) dr d\theta \\
&= \int_0^\pi \sin \theta d\theta \cdot \int_0^3 r^2 dr\\
&= (\cos \pi - \cos 0)(\frac{27}{3})\\
&= -18
\end{align*}
\end{ex}

\begin{ex}
  Another example: We plot over the region
  \sageplot[height=5cm]{polygon([[0,0], [0, 2], [1,2]], color='red')}
  \begin{align*}
    \int_{x=0}^{x=1}\int_{y=2x}^{y=2} x dy dx \\
    &= \int_{\theta = \arctan 2}^{\theta = \pi/2} \int_{r=0}^{r=2/\sin \theta} r^2 \cos \theta dr d\theta\\
  \end{align*}

  If you want $r$ to be the outer, then it's a bit harder
  We split it into 2 regions
  \sageplot[height=5cm]{polygon([[0,0], [0, 2], [1,2]], color='red') + plot(sqrt(4-x^2), (x, 0, 2))}
  \begin{align*}
    \int_{r=0}^{r=2} \int_{\theta = \arctan 2}^{\theta = \pi/2} r^2 \cos \theta d\theta dr
    + \int_{r=2}^{r=\sqrt{5}} \int_{\theta = \arctan 2}^{\theta = \arcsin 2/r} r^2 \cos \theta d\theta dr
  \end{align*}
\end{ex}
\subsubsection{Polar--Rectangular Conversions}
\[x = r\cos \theta\]
\[y=r\sin\theta\]
\[x^2+y^2=r^2\]
\[\theta = \arctan \frac{y}{x}\]
\[\Delta A \approx r \Delta \theta \Delta r\]
\[dA = r\, d\theta dr = r \, dr d\theta\]
\[\int_R f dA = \int_\alpha^\beta\int_a^b f(r\cos\theta, r\sin \theta) drd\theta\]
\section{Chapter 16 Knowledge Check Practice}
\begin{enumerate}
  \item  Consider the double integral $\int_R \delta(x,y) dA$ where $\delta(x,y)$ is the distance from $(x,y)$ to $(0,0)$ and $R$ is the region bounded by the y-axis, the line $y=1$, the line $x=1$ and the semi-cirle $y=\sqrt{5-x^2}$
    \renewcommand{\theenumi}{\Alph{enumi}}
  \begin{enumerate}
    \item Draw $R$\\
      Solution: \sageplot[height=5cm]{region_plot(y<=sqrt(5-x^2), (x, 0, 1), (y, 1,3))}
    \item find $\delta(x,y)$.\\
      Solution: $\delta(x,y) = \sqrt{x^2+y^2}$
    \item What is the practical meaning of the double integral\\
      Solution: the total mass of the region $R$.
    \item Write the double integral of the form $dxdy$. Do not evaluate.\\
      Solution: \[\int_{y=1}^{y=2} \int_{x=0}^{x=1} \sqrt{x^2+y^2} \, dx dy + \int_{y=2}^{y=\sqrt{5}} \int_{x=0}^{x=\sqrt{5-y^2}} \sqrt{x^2+y^2} \, dx dy\]
    \item Write the double integral of the form $dydx$. Do not evaluate.\\
      Solution: \[\int_{x=0}^{x=1} \int_{y=1}^{y=\sqrt{5-x^2}} \sqrt{x^2+y^2} \, dy dx\]
    \item Write the double integral of the form $drd\theta$. Do not evaluate.\\
      Solution: \[\int_{\theta=\pi/4}^{\theta=\arctan(2)} \int_{r=\sec\theta}^{r=\csc\theta} r^2 dr d\theta + \int_{\theta=\arctan 2}^{\theta=\pi/2} \int_{r=\sec\theta}^{r=\sqrt{5}} r^2 dr d\theta\]
      \item Write the double integral; of the form $d\theta dr$. Do not evaluate.\\
        Solution: \[\int_{r=1}^{r=\sqrt{2}} \int_{\theta=\arccos(1/r)}^{\theta=\pi/2} r^2 d\theta dr + \int_{r=\sqrt{2}}^{r=\sqrt{5}} \int_{\theta=\arcsin(1/r)}^{\theta=\pi/2} r^2 d\theta dr\]
      \item Set up an integral to express the average mass density of $r$\\
        Solution: \[\frac{\int_{x=0}^{x=1} \int_{y=1}^{y=\sqrt{5-x^2}} \sqrt{x^2+y^2} \, dy dx}{\int_{x=0}^{x=1} \int_{y=1}^{y=\sqrt{5-x^2}} 1 \, dy dx}\]
  \end{enumerate}
  \item A solid region $W$ is bounded above by $z=5-x^2-y^2$ and below by $z=1$.
    \begin{enumerate}
      \item Sketch $W$ in $R^3$, and the projection of $W$ onto the $xy$-plane  \\
        Solution: \sageplot[height=5cm]{implicit_plot3d(z==5-x^2-y^2, (x,-2,2 ), (y, -2,2), (z, 1, 5), region=(lambda x,y,z: z >=1))}
        and
        \sageplot[height=5cm]{implicit_plot(x^2+y^2==4, (x, -2, 2), (y, -2, 2))}
    \end{enumerate}
  \item Reverse the order of the integral
    \[\int_{x=-1}^{x=0} \int_{y=1}^{y=4-x} f(x,y) dy dx\]\\
    Solution:
    The first step is always to draw a picture of the region.
    \sageplot[height=5cm]{region_plot(y<=4-x, (x, -1, 0), (y, 1, 5))}
    \[\int_{y=1}^{y=4} \int_{x=-1}^{x=0} f(x,y) dx dy + \int_{y=4}^{y=5} \int_{x=-1}^{x=4-y} f(x,y) dx dy\]
\end{enumerate}
\section{January 23, 2023}
\begin{ex}
  Consider the region
  \sageplot[height=7cm]{region_plot(y<=4, (x, -1, 2), (y, -3, 4))}
  The limits of the integral
  \[\int_{x=-1}^{x=2} \int_{y=-3}^{y=4} f dy dx\]
  are constant because we are using CARTESIAN coordinates.
  But, by contrast, consider
  a semicircle. This will have constant limits of integration in POLAR coordaintes
\end{ex}
\begin{ex}
  Consider the region
  \sageplot[height=7cm]{region_plot(y<=2, (x, -1, 0), (y, 0, 2)) + region_plot(y<=sqrt(4-x^2), (x, 0, 2), (y, 0,2))}
  Do 4 different orders of integration
  First, $dydx$
  \[\int_{x=-1}^{x=0} \int_{y=0}^{y=2}x  \, dydx + \int_{x=0}^{x=2} \int_{y=0}^{y=\sqrt{4-x^2}} x dy dx\]
  Next $dx dy$
  \[\int_{y=0}^{y=2} \int_{x=-1}^{x=\sqrt{4-y^2}} x \,dx dy\]
  Next $dr d\theta$
  \[\int_{\theta = 0}^{\theta = \pi/2} \int_{r=0}^{r=2} r^2 \cos \theta dr d\theta +
  \int_{\theta = \pi/2}^{\theta = \pi/2 + \arctan(1/2)} \int_{r=0}^{r=2\csc \theta} r^2\cos\theta dr d\theta +
  \int_{\theta = \pi/2 + \arctan(1/2)}^{\theta = \pi} \int_{r=0}^{r=-\sec\theta} r^2\cos \theta dr d\theta \]
  Finally $d\theta dr$
  \[
  \int_{r=0}^{r=1} \int_{\theta = 0}^{\theta = \pi} r^2 \cos \theta d\theta dr
  + \int_{r=1}^{r=2} \int_{\theta=0}^{\theta = \arccos(-1/r)} r^2\cos\theta d\theta dr
  + \int_{r=2}^{r=\sqrt{5}} \int_{\theta=\arcsin(2/r)}^{\theta=\arccos(-1/r)} r^2\cos\theta d\theta dr.\]
\end{ex}
By the way, our first knowledge check on chapter 16  is on 01/26
\begin{ex}
  Imagine a circular dinner plate with radius 10cm where the mass density of the dinner plate is given by
  \[\delta(x,y) = \sqrt{x^2+y^2}\]
  use polar coordinates to find the total mass of the plate.\\
  Solution:
\begin{align*}
  \int_{\theta  = 0 } ^ {\theta = 2\pi} \int_{r=0}^{r=10} r^2 dr d\theta \\
  &= \int_{\theta = 0}^{\theta = 2\pi} 1000/3 d\theta \\
  &= 2\pi*1000/3 = \frac{2000\pi}{3}
\end{align*}
\end{ex}
\begin{ex}
  What is
  \[\int_{-\infty}^{\infty} e^{-x^2} dx?\]
  Solution:
  Let
  \[I = \int_{-\infty}^{\infty} e^{-x^2}dx\]
  Then
\begin{align*}
  I^2 &= \int_{-\infty}^{\infty} e^{-x^2}dx\int_{-\infty}^{\infty} e^{-y^2}dy \\
  &= \int_{x=-\infty}^{x=\infty} \int_{y=-\infty}^{y=\infty} e^{-(x^2+y^2)} dy dx \\
  &= \int_{\theta = 0}^{\theta = 2\pi} \int_{r=0}^{r=\infty} e^{-r^2}r dr d\theta\\
  &= \pi
  \end{align*}
So $I=\sqrt{\pi}$
\end{ex}
\subsection{16.5 - Cylindrical and Spherical Coordinates}

\subsubsection{Cylindrical Coordinates}
Cylindrical coordinates are like polar coordinates in 3D. It uses 1 angle and 2 lengths.
They are described by $(r, \theta, z)$
The conversion is
Rectangular to cylindrical:
\[(x,y,z) \to (\sqrt{x^2+y^2}, \arctan(y/x), z)\]
cylindrical to rectangular:
\[(r,\theta, z) \to (r\cos \theta, r\sin \theta, z)\]
The reason they are called cylindrical coordiantes is because if you are trying to integrate a cylinder, the bounds of integration
are constant
\subsubsection{Spherical Coordaintes}
Spherical Coordinates use 2 angles and 1 length. The coordinates are of the form $(\rho, \theta, \phi)$
The conversions are
\begin{align*}
  \rho^2 &= x^2+y^2+z^2\\
  x &= \rho\sin\phi\cos\theta\\
  y &=\rho\sin\phi\sin\theta\\
  z &= \rho\cos\phi\\
  r &=  \rho\sin\phi\\
  \phi &= \arccos(\frac{z}{\sqrt{x^2+y^2+z^2}}) \\
  dV &= \rho^2 \sin\phi d\rho d\phi d\theta
\end{align*}
\section{January 25, 2023}
\begin{ex}
  Find the volume of this cone using all integration methods.
  \sageplot[height=10cm]{implicit_plot3d(z^2 == x^2+y^2, (x, -2, 2), (y, -2, 2), (z, 0, 2))}\\
  Solution:
  Note that the equation for this cone is $x^2+y^2 = z^2$
  \[\int_{x=-2}^{x=2}, \int_{y=-\sqrt{4-x^2}}^{y=\sqrt{4-x^2}} 2-\sqrt{x^2+y^2} dy dx\]
  \[\int_{\theta=0}^{\theta = 2\pi} \int_{r=0}^{r=2} r(2-r) dr d\theta\]
  \[\int_{\theta = 0}^{\theta = 2\pi} \int_{r=0}^{r=2}\int_{z=r}^{z=2} r dz dr d\theta\]
  \[\int_{x=-2}^{x=2} \int_{y=-\sqrt{4-x^2}}^{\sqrt{4-x^2}} \int_{z=\sqrt{x^2+y^2}}^{z=2} 1 dz dy dx\]
  \[\int_{\theta = 0}^{\theta = 2\pi} \int_{\phi =0 }^{\phi = \pi/4} \int_{\rho = 0}^{\rho = 2\csc\phi} \rho^2 \sin\phi d\rho d\phi d\theta\]

\end{ex}

\begin{ex}
  Consider the 3d solid which is a dome on the bottom and  a
  cylinder on the top.
  \sageplot[height=10cm]{
    plot3d(x^2+y^2, (x,-1, 1), (y, -1, 1))
    + implicit_plot3d(x^2 + y^2 ==1, (x, -1,1), (y, -1,1), (z, 1, 2))
    + plot3d(2, (x, -1,1), (y, -1, 1))}
  Find the mass given that the mass density for a point is given by the
  distance from $z=2$.
  \[\int_{\theta = 0}^{\theta = 2\pi} \int_{r=0}^{r=2} \int_{z=r }^{z = 2} (2-z)r dz dr d\theta\]
\begin{align*}
  &\int_{\theta = 0}^{\theta = 2\pi} \int_{\phi = 0} ^{\phi = \arctan 1/2} \int_{\rho = 0}^{\rho = 2\sec\phi} \rho^2 \sin\phi d\rho d\phi d\theta\\
  &+\int_{\theta = 0}^{\theta = 2\pi} \int_{\phi = 2\csc\theta} ^{\phi = \pi/2} \int_{\rho =0 }^{\rho = \csc\phi}(2-\rho\sin\phi) \rho^2 \sin\phi d\rho d\phi d\theta\\
  &+\int_{\theta = 0}^{\theta = 2\pi} \int_{\phi = \arctan 1} ^{\phi = \pi/2} \int_{\rho = 0}^{\rho = \cos\phi/\sin^2\phi} (2-\rho\sin\phi)\rho^2 \sin\phi d\rho d\phi d\theta\\
\end{align*}
  \[\int_{x=-1}^{x=1} \int_{y= -\sqrt{4-x^2}}^{y=\sqrt{4-x^2}} \int_{z=x^2+y^2}^{z=2} 2-z dz dy dx\]
  The spherical integral seems complicated, so you should break up the shape.
  \sageplot[height=7cm]{region_plot(y >= x^2, (x, -1, 1), (y, 0, 1)) + region_plot(y<=2, (x, -1, 1), (y, 1, 2))}
\end{ex}
\begin{ex}
  Write an integral for $f$
  for the region
  \sageplot[height=8cm]{plot3d(1-sqrt(x^2+y^2), (x, -1, 1), (y, -1, 1)) + plot3d(-1+x^2+y^2, (x, -1, 1), (y, -1, 1))}
  which is bounded by $z = 1-\sqrt{x^2+y^2}$ and $z = -1+x^2+y^2$
  \[\int_{\theta = 0}^{\theta = 2\pi} \int_{r=0}^{r=1} \int_{z= r^2-1}^{1-r} fr \, dz dr d\theta \]
  \[\int_{x=-1}^{x=1} \int_{y=-\sqrt{1-x^2}}^{y=\sqrt{1-x^2}} \int_{z=-1+x^2+y^2}^{1-\sqrt{x^2+y^2}} f\, dz dy dx\]
  Remember that $z=1-\sqrt{x^2+y^2} \implies \rho = \frac{1}{\cos\phi + \sin\phi}$
  \[\int_{\theta = 0}^{\theta = 2\pi} \int_{\phi = 3\pi/4}^{\phi=\pi} \int_{\rho = 0}^{\rho = }\]
\end{ex}
\section{Knowledge Check Solutions}
Consider the double integral
\[\int_{1}^{2} \int_{1}^{\sqrt{9-x^{2}}} \delta(x,y) dA\]
where $\delta(x,y)$ is the distance from the y-axis
\begin{enumerate}
  \item Sketch $R$ \\
        \sageplot[height=7cm]{region_plot([y<sqrt(9-x^2), y > 1], (x, 1, 2), (y, 1, 3))}
  \item find $\delta(x,y)$ \\
        Solution: $\delta(x,y) = x$
  \item What is the practical meaning of the integral?\\
        Solution: The total mass
  \item dydx
        \[\int_{1}^{2}\int_{1}^{\sqrt{9-x^{2}}} x dy dx\]
  \item dx dy
    \[\int_{1}^{\sqrt{5}} \int_{1}^{2} x dx dy + \int_{\sqrt{5}}^{\sqrt{8}} \int_{1}^{\sqrt(9-y^{2})} x dx dy \]
  \item $dr d\theta$
        \[\int_{\arctan(1/2)}^{\arctan(1/1)} \int_{1/\sin\theta}^{2/\cos\theta} r^{2}\cos\theta dr d\theta + \int_{\arctan(1)}^{\arctan(\sqrt{5}/2)} \int_{1/\cos\theta}^{2/\cos\theta} r^{2}\cos\theta dr d\theta
        + \int_{\arctan(\sqrt{5}/2)}^{\arctan(\sqrt{5})} \int_{1/\cos\theta}^{3} r^{2}\cos\theta drd\theta\]

  \item $d\theta dr$
        \[\int_{\sqrt{2}}^{\sqrt{5}} \int_{{\arcsin(1/r)}}^{\arccos(1/r)} r^{2}\cos\theta d\theta dr + \int_{\sqrt{5}}^{3} \int_{\arccos(2/r)}^{\arccos(1/r)} r^{2} \cos\theta d\theta dr\]
  \item Let $W$ be the solid defined by $z \le 8-x^{2}-y^{2}$, $x^{2 } + y^{2} \le 4, z \ge -3$. Sketch $W$.
  \item dydx \\
        Solution: \[\int_{-2}^{2} \int_{\sqrt{4-x^{2}}}^{\sqrt{4-x^{2}}} (11-x^{2}-y^{2})\]
  \item $drd\theta$ \\
        Solution: \[\int_{0}^{2\pi} \int_{0}^{2} r(11-r^{2}) drd\theta\]
  \item dzdydx \\
        Solution: \[\int_{-2}^{2}\int_{-\sqrt{4-x^{2}}}^{\sqrt{4-x^{2}}} \int_{-3}^{8-x^{2}-y^{2}} 1 dz dy dx\]
  \item $dzdrd\theta$ \\
        Solution: \[ \int_{0}^{2\pi} \int_{0}^{2} \int_{-3}^{8-r^{2}} r dzdrd\theta\]
  \item $d\rho d\phi d\theta$ \\
        Solution :
\begin{align*}
        &\int_{0}^{2\pi} \int_{0}^{\arctan(2/4)} \int_{0}^{\frac{-\cos\phi + \sqrt{\cos^{2}  \phi + 32\sin^{2}}}{2\sin^{2} \phi}} \rho^{2} \sin\phi d\rho d\phi d\theta\\
        &+ \int_{0}^{2\pi} \int_{\arctan(2/4)}^{\pi - \arctan(2/3)} \int_{{0}}^{2/\sin\phi} \rho^{2} \sin\phi d\rho d\phi d\theta\\
  &+ \int_{0}^{2\pi} + \int_{\pi-\arctan(2/3)}^{\pi} \int_{0}^{-3/\cos\phi} \rho^{2} \sin\phi d\rho d\phi d\theta\\
    \end{align*}
  \item Use cylindrical coordiantes to compute the total mass if the mass density is the distance from the origin \\
        Solution \[\int_{0}^{2\pi} \int_{0}^{2} \int_{-3}^{8-x^{2}-y^{2}} r (r^{2} + z^{2}) dz dr d\theta\]
\end{enumerate}
\section{January 30, 2023 - Vector Fields}
\begin{ex}
Remember  that you can parameterize the graph of a circle by
\[\vec{r}(t) = \langle a\cos t, a \sin t \rangle , 0 \le t \le 2\pi`\]
If we add the angular frequency $\omega$, we get a more generalized form
\[\vec{r}(t) = \langle a\cos \omega t, a \sin \omega t \rangle , 0 \le t \le \frac{2\pi}{\omega}`\]
And the speed of this particle is $a \omega$

Remeber that $\vec{v} = \frac{d\vec{r}}{dt}$,
so
\[\vec{v}(t) = \langle -a\omega \sin(\omega t), - a \omega \cos (\omega t) \rangle\]
so $\|\vec{v}\| = a \omega$

Let's take the derivative again to get
\[\vec{a}(t)  = \langle -a \omega^{2} \cos(\omega t), -a \omega^{2} \sin(\omega t) \rangle\]

Note that $\vec{r}$ is always perpendicular to $\vec{v}$ because
$\vec{r} \cdot \vec{v} = 0$. Also note that $\vec{a}  = -\omega^{2} \vec{r}$.

Don't overgeneralize: $\vec{v}$ is not always perpendicular to $\vec{r}$, but it is perpendicular in circular motion.
\end{ex}
\begin{ex}
  Consider the following graph
  \sageplot[height=6cm]{parametric_plot((2*cos(x), 2*sin(x)), (x, 0, pi))}
  Which is traced out by a particle over 5 seconds. Find the location of the particle at time $t$.\\
  Solution: The equation of the particle is
  \[\vec{r}(t) = \langle 2 \cos(-\pi t/5), 2 \sin (-\pi t/5) \rangle\]
  You can also paramtetrize $(r, \theta)$ by $t$.
  \[r = 2, \qquad\theta = \pi - \pi t/5\]
\end{ex}

\begin{ex}
  A particle moves in a circle of radius $2$ at $5 m/s$. Find the equation of the particle. \\
  Solution:
  \[\vec{r} = \langle 2 \sin (5 t /2) , 2 -\cos (5t / 2)\rangle,  0 \le t \le \pi / 5 \]

\end{ex}
\begin{defn}
  A \textbf{vector field} is a function from $\mathbb{R}^{n} \to \mathbb{R}^{n}$. It takes in a vector and spits out a vector.  For example,
  \[F(x,y) = \langle x + y, x - y \rangle\]
\end{defn}
From now on, everything is about vector fields.

The way to sketch a vector field in $\mathbb{R}^{2}$ is to draw a little arrow at each point representing the output vector.
\begin{ex}
  For example, here is a sketch of
$F(x,y) = \langle x + y, 2x - y \rangle$\\
\sageplot[height=5cm]{plot_vector_field((x + y, 2*x  - y), (x, -3, 3), (y, -3, 3))}
\end{ex}

\begin{ex}
  sketch $F(x,y) = \langle x, y \rangle$ and $F(x,y)  = \langle -y, x \rangle$?
  Solution:\\
\sageplot[height=5cm]{plot_vector_field((x,y), (x, -3, 3), (y, -3, 3))}\\
\sageplot[height=5cm]{plot_vector_field((-y, x), (x, -3, 3), (y, -3, 3))}\\
\end{ex}
\section{Februrary 1 2023}
\subsection{17.3 - Flow Line}
\begin{defn}
  The \textbf{flow line} is
  how a particle in a vector field would ``flow'' (imagine the vector field is a force exerted on the particle).
  A path $\sigma(t) : \mathbb{R} \to \mathbb{R}^{n}$ is a flow-line in a vector field $\vec{F}(x) : \mathbb{R}^{n} \to \mathbb{R}^{n} $ iff
  \[\sigma'(t) = \vec{F}(\sigma(t))\]
\end{defn}

Let's review parametric functions
\begin{ex}
  Find the flow line defined by $\vec{r}(t)$
  if the vector field is $\vec{F} = \langle 1,3 \rangle$ and  $\vec{r}(0) = \langle 2, 4 \rangle$\\
  Solution:
  \[\vec{r}(t) = \langle 2 + t, 4 + 3t \rangle \]
\end{ex}
\begin{ex}
  Find the flow line of $\vec{r}(t)$ if
  the vector field is $\vec{F}(x,y) = \langle y, 2y \rangle$
  and $\vec{r}(1) = \langle 3,4 \rangle$\\
  Solution:
  \begin{align*}
    r' &= F \circ r \\
    \langle x', y' \rangle &= \langle y, 2y \rangle \\
\vec{r}(t) &= \langle 2 \, e^{\left(2 \, t\right)} + 1, 4 \, e^{\left(2 \, t\right)}
\rangle
  \end{align*}
\end{ex}
Note that you can find flow lines in sage using
\begin{sageblock}
# I am going to do this later
print('hello world')
\end{sageblock}

\begin{ex}
  Find the flowline if the field is $\vec{F}(x,y) = \langle 2y, 1 \rangle$
  and $\vec{r}(0) = \langle 3, 4 \rangle$ \\
  Solution:\\
  Note that a sketch can help you find the solution\\
\sageplot[height=5cm]{plot_vector_field((2*y, 1), (x, -3, 3), (y, -3, 3))}\\
  \begin{align*}
    x' = 2y &, y' = 1, x(0) = 3, y(0) = 4 \\
    y &= t +  4\\
    x &= t^{2}  + 8t +  3\\
    \vec{r}(t) &= \langle t^{2} + 3, t  + 4\rangle
  \end{align*}
\end{ex}
\begin{ex}
  Consider the vector field $\vec{F}(x,y) = \langle y, x \rangle$ and $\vec{r}(t) = \langle 3, 4 \rangle$
  Solution: \\
  \[x\left(t\right) = -\frac{1}{2} \, e^{\left(-t\right)} + \frac{7}{2} \, e^{t}, y\left(t\right) = \frac{1}{2} \, e^{\left(-t\right)} + \frac{7}{2} \, e^{t}
\]
\end{ex}
\subsection{17.4 -- Euler's Method}
If we cannot find an exact solution, we can do a numerical approximation using Euler's Method (approximate differential equations with a tangent line).
The main idea is that
\[f(a + \Delta x) \approx f(a) + \Delta x f'(a)\]
\section{Chapter 17 Practice Knowledge Check}
\begin{enumerate}
  \item A particle moves in the direciton of $\langle 4,3 \rangle $ along a straight line at a constant speed of 10 and is located at $(1,2)$ and $(a, 0)$ when $t=k$
        and $t=5$ respectively. Find $\vec{r}$
        Solution: \[\vec{r}(t) = \langle -31 + 8t, -22 + 6t \rangle\]
  \item A particle is located at $(0, -4)$ and moves along a full circlular path clockwise centered at the origin at a cosntant
        speed of $\pi$. Find the particle's veclocity at $t=2$\\
        Solution: \[4\pi\]
        (note that this question was dumb and $\pi$ referred to )
  \item let $\vec{F}(x,y) = \langle 2, 4y \rangle$ be ocean current. An ice berg is at $(3,5)$ at $t=0$.
        \begin{enumerate}
          \item Use Euler's Method with two steps to approximate the location of the iceberg at t=1. \\
                Solution:
                \begin{align*}
                    r(0) &= \langle 3, 5\rangle \\
                  r(1/2) &\approx \langle 3, r \rangle + 1/2 \rangle 2, 4 * 5 \rangle = \langle 4, 15 \rangle \\
                  r(1) &\approx \langle 4, 15 \rangle + 1/2 \langle 2, 4*15 \rangle = \langle \rangle
                  \end{align*}
          \item Find the exact location of the iceberg in the ocean current at $t=1$\\
                Solution:
                \begin{align*}
                  x' = 2&, y' = 4y \\
                  x = 2t + 3&, y = 5e^{4t} \\
                  r(1) &= \langle 5, 5e^{4} \rangle
                \end{align*}
          \item Sketch the vector field and draw the flow line starting at $(3,5)$ on the vector field \\
                Solution:
                \sageplot[height=7cm]{plot_vector_field((2, 4*y), (x, 0, 10), (y, 0, 10)) + streamline_plot((2, 4*y), (x, 0, 10), (y, 0, 10), start_points=[[3,5]])}
        \end{enumerate}
\end{enumerate}
\section{February 2 2023}
\subsection{18 -- Line Integral}
A line integral can be motivated by considering work from physics. In physics, $W = \vec{F} \cdot \Delta\vec{ x}$.
\begin{defn}
  Given a vector field $\vec{F} : \mathbb{R}^{n} \to \mathbb{R}^{n}$ and a curve $C$ parametrized by the parametric equation
  $\vec{r}(t) : \mathbb{R} \to \mathbb{R}^{n}$, then the work done by the field along the curve is given by the \textbf{line integral}
  \[\int_{C} \vec{F} \cdot d\vec{r} = \int_{a}^{b} \vec{F}(\vec{r}(t)) \cdot \vec{r} '(t) dt\]
\end{defn}
\begin{ex}
  Given the vector field $\vec{F} = \langle 3, 4 \rangle$ and the path $C$ which is a line from
  $(0, 0)$ to $(0, 5)$, find the work done. \\
  Solution:
  Let $\vec{r}(t) = \langle 0, t \rangle, 0 \le t \le 5$.
  \[\int_{C} \vec{F} \cdot d\vec{r} = \int_{0}^{5} \langle 3, 4\rangle \cdot \langle 0, 1 \rangle dt
   = 20\]
\end{ex}
\begin{ex}
  Find the work done by the vector field $\vec{F}(x,y) = \langle x+ y, x \rangle$
  along the curve\\
  \sageplot[height=5cm]{plot(2*x+2, (x, -1, 0)) + plot(sqrt(4-x^2), (x, 0, 2)) + plot_vector_field((x+y, x), (x, -1, 2), (y, 0, 2))}\\
  solution: \\
  let $r_{1}(t) = \langle t,  2t + 2 \rangle, -1 \le t \le 0$
  and $r_{2}(t) = \langle 2\sin t, 2\cos t \rangle, 0 \le t \le \pi/2 $
  Then
\begin{align*}
  \int_{-1}^{0} \langle 3t + 2, t \rangle \cdot \langle 1, 2 \rangle dt
  + \int_{0}^{\pi/2} \langle 2 \sin t + 2 \cos t, 2 \sin t \rangle \cdot \langle 2 \cos t, -2 \sin t \rangle dt\\
  \int_{-1}^{0} 5t + 2 dt + \int_{0}^{2} 4 \sin t \cos t + 4 \cos^{2} t - 4 \sin^{2} t dt \\
  -(\frac{5}{2} + 2) + (1 + 1 ) = \frac{3}{2}
\end{align*}

\end{ex}
\begin{ex}
  Given the vector field $\vec{F}(x,y) = \langle 2 , x \rangle$ and the curve $C$ which is a parabola given below\\
  \sageplot[height=5cm]{plot(x^2, (x, -1, 2)) + plot_vector_field((2, x), (x, -1, 2), (y, 0, 4))}\\
  Solution: \\
  Let $\vec{r}(t) = \langle t, t^{2} \rangle, -1 \le t \le 2$
  then
  \begin{align*}
    \int_{-1}^{2} \langle 2, t \rangle \cdot \langle 1, 2t \rangle dt \\
    &= 12
  \end{align*}
\end{ex}

\section{Februrary 6 2023}
YHPL has a meeting. Maybe no class.
\end{document}
